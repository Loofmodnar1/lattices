% Copyright 2013 Nicolai Hähnle <nhaehnle@gmail.com>
%
% This work is licensed under the Creative Commons Attribution-ShareAlike 3.0
% Unported License, see http://creativecommons.org/licenses/by-sa/3.0/
%
% Among other things, this means that yes, you may take e.g. illustrations from
% the book and use them in your own work. However, (a) you must give proper
% attribution by naming me as its original author and (b) you must make your
% derivative work available under the same or similar license terms.
%
% See the Creative Commons website for the exact licensing terms.

\documentclass[a4paper,10pt]{scrbook}
\usepackage[utf8]{inputenc}

\usepackage{amsmath}
\usepackage{amssymb}
\usepackage{amsthm}
\usepackage{clrscode}
\usepackage{multirow}
\usepackage{enumerate}
\usepackage{tikz}

\usetikzlibrary{calc}
\usetikzlibrary{intersections}
\usetikzlibrary{arrows}
\usetikzlibrary{decorations.pathmorphing}

\newcommand{\N}{\mathbb{N}}
\newcommand{\Q}{\mathbb{Q}}
\newcommand{\R}{\mathbb{R}}
\newcommand{\Z}{\mathbb{Z}}
\newcommand{\C}{\mathbb{C}}
\newcommand{\cA}{\mathcal{A}}
\newcommand{\cP}{\mathcal{P}}
\newcommand{\cV}{\mathcal{V}}
\providecommand{\one}{\mathbf{1}}
\DeclareMathOperator{\vol}{vol}
\DeclareMathOperator{\cone}{cone}
\DeclareMathOperator{\conv}{conv}
\DeclareMathOperator{\diam}{diam}
\DeclareMathOperator{\poly}{poly}
\DeclareMathOperator{\Ker}{Ker}

\theoremstyle{plain}
\newtheorem{theorem}{Theorem}[chapter]
\newtheorem{lemma}[theorem]{Lemma}
\newtheorem{proposition}[theorem]{Proposition}
\newtheorem{claim}[theorem]{Claim}
\newtheorem{corollary}[theorem]{Corollary}
\newtheorem{fact}[theorem]{Fact}
\newtheorem{conjecture}{Conjecture}

\theoremstyle{definition}
\newtheorem{definition}[theorem]{Definition}
\newtheorem{notation}[theorem]{Notation}
\newtheorem{example}[theorem]{Example}
\newtheorem{remark}[theorem]{Remark}
\newtheorem{problem}[theorem]{Problem}

\usepackage[colorlinks]{hyperref}

%opening
\title{Lattices and Convex Bodies}
\author{Nicolai Hähnle}

%\includeonly{chapter04-dual-fourier}

\begin{document}

\maketitle

\tableofcontents

\chapter*{Preface}

These notes are being prepared as part of a lecture ``Lattices and Convex Bodies''
given by Nicolai Hähnle at the University of Bonn in Fall 2013.

The most recent version can be found at \url{https://github.com/nhaehnle/lattices}
in the form of a Git repository of the \LaTeX{} sources.

This work is licensed under the Creative Commons Attribution-ShareAlike 3.0
Unported License (CC BY-SA 3.0). See \url{http://creativecommons.org/licenses/by-sa/3.0/}
for detailed information about what this means.

Contributions of spotted errors, improved proofs, and so on are welcome,
and will be added with proper attribution.
See the Github site linked above for more information.


\section*{Contributions}

Typos and other corrections have been contributed by:
\begin{quote}
  Rasmus Schröder
\end{quote}


% Copyright 2013 Nicolai Hähnle <nhaehnle@gmail.com>
%
% This work is licensed under the Creative Commons Attribution-ShareAlike 3.0
% Unported License, see http://creativecommons.org/licenses/by-sa/3.0/
%
% Among other things, this means that yes, you may take e.g. illustrations from
% the book and use them in your own work. However, (a) you must give proper
% attribution by naming me as its original author and (b) you must make your
% derivative work available under the same or similar license terms.
%
% See the Creative Commons website for the exact licensing terms.

\chapter{Lattice Basics and Minkowski's theorem}
\label{chapter:basics}

Let us begin with an old question:
\begin{problem}
  When can a natural number $n$ be expressed as the sum of two squares, that is,
  when can we write $n = x^2 + y^2$ where $x$ and $y$ are integers?
\end{problem}
Suppose that $n = ab$, where both $a$ and $b$ are sums of two squares of integers, that is
\begin{align*}
  a &= x^2 + y^2 \\
  b &= z^2 + w^2
\end{align*}
Then
\begin{align*}
  n &= ab \\
    &= (x^2 + y^2)(z^2 + w^2) \\
    &= x^2 z^2 + x^2 w^2 + y^2 z^2 + y^2 w^2 \\
    &= (xz + yw)^2 + (xw - yz)^2
\end{align*}
That is, every product of sums of two squares can be written as a sum of two squares.
This suggests we should focus on indivisible factors, i.e. prime numbers.

\begin{problem}
  \label{problem:prime-sum-of-squares}
  When can a prime $p$ be expressed as the sum of two squares, that is,
  when can we write $p = x^2 + y^2$ where $x$ and $y$ are integers?
\end{problem}
The sum of squares reminds us of Pythagoras' theorem:
\begin{center}
  \begin{tikzpicture}
    \draw (0,0) -- node[below] {$x$} (3,0) -- node[right] {$y$} (3,2)-- node[above] {$\sqrt{p}$} (0,0);
  \end{tikzpicture}
\end{center}
We can rephrase Problem~\ref{problem:prime-sum-of-squares}:
For which primes $p$ is there an integer point on the circle of radius $\sqrt{p}$ around the origin?

Geometry alone cannot answer this question. Let us look at some examples:
\[ \mathbf{2}, 3, \mathbf{5}, 7, 11, \mathbf{13}, \mathbf{17}, 19, 23, \mathbf{29}, 31, \mathbf{37}, \mathbf{41}, 43, 47, \mathbf{53}, 59, \mathbf{61}, \dots \]
Ignoring the special case of the rather odd prime $2$,
the primes that can be written as a sum of two squares appear to be exactly those that are congruent to $1$ modulo $4$.
Indeed, $x^2 \equiv 0$ or $1 \pmod{4}$ for all integers $x$, so $x^2 + y^2 \equiv 0, 1, 2 \pmod{4}$.
\begin{fact}
  If $p \equiv 3 \pmod{4}$, then $p$ cannot be written as the sum of two squares.
\end{fact}
Let us consider a different modular arithmetic angle.
Suppose we have $p = x^2 + y^2$, then certainly $x^2 + y^2 \equiv 0 \pmod{p}$.
Since $p$ is a prime, $\Z / (p)$ is a field and we can rearrange to get
\[ (xy^{-1})^2 \equiv -1 \pmod{p}. \]
That is, we have a square root of $-1$.
Algebra tells us exactly when the field $\Z/(p)$ contains such a square root:
$(\Z/(p))^\star$ is cyclic of order $p-1$, so there exists an element $q$ of order $4$ if and only if $4 | (p - 1)$,
which is another way of saying that there is a square root $q$ of $-1$ if and only if $p \equiv 1 \pmod{4}$.

In this case, given any integer $a$, we have
\[ a^2 + (aq)^2 \equiv 0 \pmod{p}, \]
which means $a^2 + (aq)^2$ is -- if not equal to $p$ -- at least a multiple of $p$.
Can we use this observation to express $p$ as the sum of squares of two integers?

We certainly want both integers to be in the range $\{ 1, \dots, p - 1 \}$.
Even if $a$ satisfies this condition, $qa$ is likely to fall outside this range.
Luckily, $a^2 + (bp + aq)^2$ is a multiple of $p$ as well, for any integer $b$.
Is there a choice of $a$ and $b$ such that $a^2 + (bp + aq)^2 = p$?
Or, to put it differently, is there a point
\[
  x =
  \underbrace{\begin{pmatrix}
    q & p \\
    1 & 0
  \end{pmatrix}}_{=: B}
  \begin{pmatrix}
    a \\ b
  \end{pmatrix},
  a, b \in \Z
\]
such that $\|x\|_2 = \sqrt{p}$?

We have returned to the \emph{geometric} formulation of our problem,
except that we reduced the set of candidate points to a \emph{proper subset} of $\Z^2$ with a very specific structure.
This chapter develops the tools that will allow us to exploit this structure and solve Problem~\ref{problem:prime-sum-of-squares}.

\section{Basic definitions}

\begin{definition}
  A \emph{lattice} $\Lambda$ is a discrete additive subgroup of $\R^d$.
  Its \emph{rank} or \emph{dimension} $\dim\Lambda$ is the dimension of the linear span of $\Lambda$.
\end{definition}

In this context, \emph{discrete} means:
for every $x \in \Lambda$, there is an $\varepsilon > 0$ such that
$\|y - x\|_2 \geq \varepsilon$ for all $y \in \Lambda \setminus \{ x \}$.

Due to the additive structure of $\Lambda$, the quantifiers can be exchanged in the previous statement.
Let $\varepsilon > 0$ be such that $\|y\|_2 \geq \varepsilon$ for all $y \in \Lambda \setminus \{ 0 \}$.
Now, is it possible that there are some $x' \neq y' \in \Lambda$ with $\|y' - x'\|_2 < \varepsilon$?
No, because $y' - x' \in \Lambda \setminus \{ 0 \}$.
From this, a simple volume packing argument shows:
\begin{lemma}
  \label{lemma:finitely-many-points-in-bounded-region}
  A bounded region of space contains only finitely many points of any lattice.
\end{lemma}
\begin{proof}
  Let $\Lambda$ be a lattice and $R > 0$.
  It is sufficient to show that the ball of radius $R$ around the origin,
  which we write as $B(0,R)$, contains only finitely many points of $\Lambda$.

  Let $\varepsilon > 0$ such that $\|y\|_2 > \varepsilon$ for all $y \in \Lambda \setminus \{ 0 \}$.
  Note that the balls $B(x,\varepsilon/2)$ and $B(y,\varepsilon/2)$ are disjoint for $x \neq y \in \Lambda$,
  see Figure~\ref{fig:finitely-many-points-in-bounded-region}.
  Therefore,
  \[
    \vol B(0,R + \varepsilon/2) \geq \vol \bigcup_{x \in B(0,R)\cap \Lambda} B(x,\varepsilon/2) = \#(B(0,R) \cap \Lambda) B(x,\varepsilon/2),
  \]
  from which it follows that $B(0,R) \cap \Lambda$ is finite.
\end{proof}
\begin{figure}
  \begin{center}
  \begin{tikzpicture}
    \foreach \a/\b in {0/0,1/0,0/1,1/1,0/2,1/2,1/-1,2/-1,1/-2,
    -1/0,0/-1,-1/-1,0/-2,-1/-2,-1/1,-2/1,-1/2
    }
      \draw[fill=black!10] ($\a*(1,0.1) + \b*(0.1,0.7)$) circle[radius=0.3cm];

    \draw[thick] (0,0) circle[radius=2cm];
    \draw[thick] (0,0) circle[radius=2.3cm];

    \clip (-3.1,-3.1) rectangle (3.1,3.1);
    \foreach \a in {-7,-6,...,7.1}
      \foreach \b in {-4,-3,...,4.1}
        \fill ($\a*(1,0.1) + \b*(0.1,0.7)$) circle[radius=2pt];

    \draw (0,0) -- node[below,near end] {$R$} (2,0);
  \end{tikzpicture}
  \end{center}
  \caption{A volume argument shows that a bounded region contains only finitely many lattice points.}
  \label{fig:finitely-many-points-in-bounded-region}
\end{figure}

\begin{corollary}
  Every lattice (except for $\Lambda = \{ 0 \}$) has a shortest non-zero vector.
\end{corollary}

Note that every lattice has at least two shortest vectors,
and we have already seen lattices like $\Z^d$ that have more shortest vectors.
We will see an upper bound on the number of shortest vectors in chapter~\ref{chapter:voronoi-cell}.

\begin{notation}
  The length of a shortest (non-zero) vector in the lattice is usually denoted as $\lambda_1(\Lambda)$
  or just $\lambda_1$ when the lattice is clear from the context.
\end{notation}



\begin{example}
  \begin{enumerate}
    \item $\Z^d$ is a lattice.

    \item Given an invertible matrix $B \in \R^{d \times d}$, the set
      \[ \Lambda(B) := \{ B t ~:~ t \in \Z^d \} \]
      is a lattice.

    \item Given rational vectors $b_1, \ldots, b_m \in \Q^d$, the set
      \[ \Lambda(b_1, \ldots, b_m) := \{ \sum_{j=1}^m t_j b_j ~:~ t_j \in \Z \} \]
      is a lattice.

    \item The condition of rationality cannot simply be dropped.
      When $d = 1$, $b_1 = 1$, $b_2 = \alpha$, where $\alpha$ is any irrational number,
      then the set $\Lambda(b_1, b_2)$ is dense in $\R$ and therefore not a lattice.

      This follows from the equidistribution theorem proved by Weyl, Sierpinski, and others.
      It can be seen in the context of Diophantine approximation, which we will see a bit of later.
  \end{enumerate}
\end{example}

\begin{definition}
  Given a lattice $\Lambda$,
  a \emph{basis} of $\Lambda$ is a linearly independent set of vectors $b_1, \ldots, b_k$
  such that $\Lambda = \Lambda(b_1, \ldots, b_k)$.
\end{definition}

We often think of basis vectors as column vectors of a matrix $B = (b_1, \ldots, b_k)$,
and write $\Lambda(B)$ for the lattice generated by $B$.
Clearly, $\dim\Lambda = k$.
We will mostly restrict our attention to full-dimensional lattices, i.e. the case $k = d$.

Every lattice has a basis; we postpone the proof of this fact until section~\ref{sec:bases}.
Except for the trivial case $d \leq 1$, a lattice has infinitely many bases,
and the choice of basis matters a great deal for computational problems.
We will discuss some related issues in chapter~\ref{chapter:basis-reduction-LLL}.

For now, just let $B$ be an invertible matrix,
which we think of as a lattice basis of the full-dimensional lattice $\Lambda = \Lambda(B)$.

Let $B'$ be another basis of $\Lambda$.
By definition, the vectors in $B'$ can be expressed as integer linear combinations of the vectors in $B$
and vice versa. Hence there are matrices $U, V \in \Z^{d \times d}$ such that
\[ B' = BU, B = B'V. \]
Taken together, this implies $V = U^{-1}$.
Since $\det$ is a group homomorphism, it follows that $\det(U) = \det(U^{-1}) = \pm 1$.

This implies that the absolute value of $\det(B)$ is an invariant of the lattice
and justifies the following definition:
\begin{definition}
  The \emph{determinant} of a full-dimensional lattice $\Lambda$
  is defined via $\det(\Lambda) := |\det(B)|$,
  where $B$ is any basis of $\Lambda$.
\end{definition}

In Section~\ref{sec:determinant-general-lattices},
we will see that this definition can be extended to general, not full-dimensional lattices.
For now, we will work only with the simpler definition given here.

\begin{definition}
  A matrix $U \in \Z^{d \times d}$ with $\det(U) = \pm 1$ is called \emph{unimodular}.
\end{definition}

\begin{lemma}
  \label{lemma:basis-exchange-is-unimodular}
  Let $\Lambda \subset \R^d$ be a lattice of dimension $k$ and let $B \in \R^{d \times k}$ be a basis of $\Lambda$.
  Then for $U \in \R^{k \times k}$ we have that $BU$ is a basis of $\Lambda$ if and only if $U$ is unimodular.
\end{lemma}
\begin{proof}
  The implication from left to right follows from the discussion above.
  For the reverse implication, one first sees that $\Lambda(BU) \subseteq \Lambda$ because $U$ is integral.
  Then we note that $U^{-1}$ is also integral because $\det(U) = \pm 1$,
  which implies that $\Lambda = \Lambda(BU\cdot U^{-1} \subseteq \Lambda(BU)$.
  Thus we have $\Lambda = \Lambda(BU)$, which completes the proof.
\end{proof}




\begin{definition}
  Let $\Lambda(B)$ be a lattice.
  The \emph{fundamental parallelepiped of $B$} is
  \[ \cP_B := \{ x = \sum_{j=1}^d \lambda_j b_j ~:~ 0 \leq \lambda_j < 1 \forall j = 1 \ldots n \}. \]
  We omit the subscript when the basis is clear from the context.
\end{definition}
See Figure~\ref{fig:fundamental-parallelepiped} for an illustration.
Observe that $\cP_B$ is the image of the half-open unit cube
under the linear transformation given by $B$. Hence
\[ \vol(\cP_B) = |\det(B)| \]

\begin{figure}
  \begin{center}
  \begin{tikzpicture}
    \clip (-2.1,-1.1) rectangle (6.1,3.1);
    \foreach \a in {-7,-6,...,7.1}
      \foreach \b in {-4,-3,...,4.1}
        \fill ($\a*(1,0.1) + \b*(0.1,0.7)$) circle[radius=2pt];

    \draw (0,0) node[below right] {$0$};

    \draw[fill=black!10] (0,0) -- (2.1,0.9) -- (3.2,1.7) -- (1.1,0.8) -- cycle;

    \draw[thick,->] (0,0) -- (2.1,0.9) node[below] {$b_2$};
    \draw[thick,->] (0,0) -- (1.1,0.8) node[above left] {$b_1$};
  \end{tikzpicture}
  \end{center}
  \caption{The fundamental parallelepiped of a lattice.}
  \label{fig:fundamental-parallelepiped}
\end{figure}



\section{Minkowski's theorem}

The power of lattices comes from studying them in relation to sets in $\R^d$.
A \emph{convex body} is a full-dimensional compact convex set.
Following the natural group theoretic definitions, we use
\[ x \equiv y \pmod{\Lambda} \]
to mean $x - y \in \Lambda$.
In this case, we say that $x$ is congruent to $y$ modulo $\Lambda$.

\begin{theorem}[Blichfeldt's theorem]
  \label{thm:blichfeldt}
  Let $\Lambda \subset \R^d$ be a lattice and $k \in \N$.
  Let $M \subseteq \R^d$ be a measurable set with $\vol(M) > k \det(\Lambda)$.
  Then there exist $k+1$ pairwise different points $x_1, \ldots, x_{k+1} \in M$
  that are congruent modulo $\Lambda$.
\end{theorem}
\begin{figure}
  \begin{center}
  \begin{tikzpicture}
    \def\M{
      \draw[thick,fill=gray,fill opacity=0.2]
        (-0.3,-0.5) -- (2.2,1.4) -- (5.9,3.1) -- (3.0,1.4) -- (3.2,1.2) -- cycle;
      \draw[thick,fill=gray,fill opacity=0.2]
        (5.2,1.9) circle[x radius=0.8,y radius=0.2,rotate=30];
    }

    \begin{scope}[shift={(0,-1)}]
      \clip (-1.1,-1.6) rectangle (6.6,3.5);
      \foreach \a in {-7,-6,...,7.1}
        \foreach \b in {-4,-3,...,4.1}
          \fill ($\a*(2,0.2) + \b*(0.2,1.4)$) circle[radius=2pt];

      \foreach \a in {-8,-7,...,8.1}
        \draw ($\a*(4.2,1.8)$) +($-9*(2.2,1.6)$) -- +($9*(2.2,1.6)$);

      \foreach \b in {-8,-7,...,8.1}
        \draw ($\b*(2.2,1.6)$) +($-9*(4.2,1.8)$) -- +($9*(4.2,1.8)$);

      \M
    \end{scope}

    \begin{scope}[shift={(6.8,0.1)}]
      \draw (0,0) -- (4.2,1.8) -- (6.4,3.4) -- (2.2,1.6) -- cycle;
      \clip (0,0) -- (4.2,1.8) -- (6.4,3.4) -- (2.2,1.6) -- cycle;

      \M
    \end{scope}

    \begin{scope}[shift={(6.8,-0.9)}]
      \draw (0,0) -- (4.2,1.8) -- (6.4,3.4) -- (2.2,1.6) -- cycle;
      \clip (0,0) -- (4.2,1.8) -- (6.4,3.4) -- (2.2,1.6) -- cycle;

      \begin{scope}[shift={(2.2,1.6)}]
        \M
      \end{scope}
    \end{scope}

    \begin{scope}[shift={(6.8,-1.9)}]
      \draw (0,0) -- (4.2,1.8) -- (6.4,3.4) -- (2.2,1.6) -- cycle;
      \clip (0,0) -- (4.2,1.8) -- (6.4,3.4) -- (2.2,1.6) -- cycle;

      \begin{scope}[shift={(-2,-0.2)}]
        \M
      \end{scope}
    \end{scope}

    \begin{scope}[shift={(6.8,-3.4)}]
      \draw (0,0) -- (4.2,1.8) -- (6.4,3.4) -- (2.2,1.6) -- cycle;
      \clip (0,0) -- (4.2,1.8) -- (6.4,3.4) -- (2.2,1.6) -- cycle;

      \foreach \a in {-7,-6,...,7.1} {
        \foreach \b in {-4,-3,...,4.1} {
          \begin{scope}[shift={($\a*(2,0.2) + \b*(0.2,1.4)$)}]
            \M
          \end{scope}
        }
      }
    \end{scope}

    \foreach \d in {-1,0,1}
      \draw[very thick,->] ($(10.0,-0.7) + \d*(0.35,0.15)$) -- +(0.0,-0.55);
  \end{tikzpicture}
  \end{center}
  \caption{Proof of Blichfeldt's theorem.}
  \label{fig:blichfeldt-theorem}
\end{figure}
\begin{proof}[Proof sketch]
  This result is best thought of as a variant of the pigeon hole principle.
  The fundamental parallelepiped $\cP$ associated to any basis of $\Lambda$
  tiles the space.
  We cut $M$ along those tiles and translate all the pieces by a lattice vector
  back to the fundamental parallelepiped,
  see Figure~\ref{fig:blichfeldt-theorem}.
  Then there must be a point in $\cP$ which is covered by at least $k+1$ of those translated pieces.
  This intuition can be formalized using integrals over characteristic functions.
\end{proof}

\begin{theorem}[Minkowski's first theorem]
  \label{thm:minkowski-first}
  Let $\Lambda$ be a lattice
  and let $M$ be a convex measurable set that is symmetric with respect to the origin
  and has $\vol(M) > 2^d \det(\Lambda)$.
  Then $M$ contains a non-zero lattice vector.

  If we replace ``convex set'' by ``convex body'',
  then $\vol(M) \geq 2^d \det(\Lambda)$ is a sufficient condition for the same result.
\end{theorem}
\begin{proof}[Proof sketch]
  Applying Blichfeldt's theorem to $\frac{1}{2} M$
  yields two points whose difference is a non-zero lattice vector in $M$.
  If $M$ is a convex body,
  then $(1 + \varepsilon) M$ contains no additional lattice vector for sufficiently small $\varepsilon > 0$,
  and we can apply the previous argument.
\end{proof}


\section{Applying Minkowski's theorem}

Let us return to Problem~\ref{problem:prime-sum-of-squares} from the beginning of the chapter.
We concluded with the question: Is there a point
\[
  x =
  \underbrace{\begin{pmatrix}
    q & p \\
    1 & 0
  \end{pmatrix}}_{=: B}
  \begin{pmatrix}
    a \\ b
  \end{pmatrix},
  a, b \in \Z
\]
such that $\|x\|_2 = \sqrt{p}$?
We can now answer this question in the affirmative.

Recall that $q \in \Z$ was chosen such that $a^2 + (bp + aq)^2$ is a multiple of $p$,
and therefore $\|x\|_2^2$ is a multiple of $p$ for every $x \in \Lambda(B)$.
It is therefore sufficient to show that the open disk around the origin with radius $\sqrt{2p}$
contains a non-zero lattice point in its interior.

Let $D$ be the disk around the origin with radius $\sqrt{2p}$, see Figure~\ref{fig:prime-sum-of-squares}.
We have $\vol(D) = 2\pi p$ and $\det \Lambda = |\det B| = p$.
Since $\vol(D) > 4 \det \Lambda$,
we can apply Minkowski's Theorem~\ref{thm:minkowski-first}
to the interior of $D$,
which completes the solution of Problem~\ref{problem:prime-sum-of-squares}:
\begin{theorem}
  Let $p$ be an odd prime.
  Then $p$ is the sum of two squares if and only if $p \equiv 1 \pmod{4}$.
\end{theorem}

\begin{figure}
  \begin{center}
  \begin{tikzpicture}
    % lattice coordinates scaled to 60%
    \clip (-4,-2.8) rectangle (5,2.8);
    \foreach \a in {-7,-6,...,7.1}
      \foreach \b in {-6,-5,...,6.1}
        \fill ($\a*(3,0) + \b*(1.8,0.6)$) circle[radius=2pt];

    \draw[thick,->] (0,0) -- (1.8,0.6) node[above] {$b_2$};
    \draw[thick,->] (0,0) -- (3,0) node[below] {$b_1$};

    \draw (0,0) node[below] {$0$};

    \draw[thick,fill=gray,fill opacity=0.2] (0,0) circle[radius=1.897];
  \end{tikzpicture}
  \end{center}
  \caption{Applying Minkowski's theorem to Problem~\ref{problem:prime-sum-of-squares} for $p = 5$ and $q = 3$.}
  \label{fig:prime-sum-of-squares}
\end{figure}


For another classic application of Minkowski's theorem,
consider the problem of Simultaneous Diophantine Approximation.
Given numbers $\alpha_1, \ldots, \alpha_n \in \R$,
we want to find a single $q \in \N$ and $p_1,\ldots, p_n \in \Z$
such that $|p_j - q\alpha_j|$ is small for all $j$.

%TODO



\section{The existence of bases}
\label{sec:bases}

We can already see that \emph{short but non-zero} lattice vectors play an important role:
representing a prime number as a sum of two squares can be achieved by finding a shortest non-zero vector in a certain lattice.
Diophantine approximations can be found as a short non-zero vectors with respect to the $\ell_\infty$ norm.
Later, we will see the relationship between short vectors and the integer programming problem.
Lattice bases play an essential role in lattice algorithms, because they are used for the numerical representation of lattices.
In this section, we will show a result that was promised earlier.

\begin{lemma}
  \label{lemma:every-lattice-has-basis}
  Every lattice $\Lambda$ has a basis.
\end{lemma}
\begin{proof}
  The proof of this theorem proceeds by induction over $d = \dim \Lambda$.
  The cases $d = 0$ and $d = 1$ are easy to see.

  Let $b_1 \in \Lambda \setminus \{ 0 \}$ be a \emph{primitive} vector,
  that is, a lattice vector such that $\lambda b_1 \not\in \Lambda$ for $\lambda \in (0,1)$
  (for example, one might take a shortest vector of the lattice).
  Let
  \begin{align*}
    \pi_1 : \R^d & \to \langle b_1 \rangle^\bot \\
    x & \mapsto x - \frac{x^T b_1}{b_1^T b_1} b_1
  \end{align*}
  be the orthogonal projection onto the orthogonal complement of the line spanned by $b_1$.
  \begin{lemma}
    $\pi_1(\Lambda)$ is a lattice.
  \end{lemma}
  \begin{proof}
    It is clear that $\pi_1(\Lambda)$ is a subgroup of $\R^d$.

    Now, note that every $x' \in \pi_1(\Lambda)$ has a pre-image $x \in \Lambda$ that satisfies
    \[
      \frac{x^T b_1}{b_1^T b_1} \in (-1/2, 1/2],
    \]
    because we can add integer multiples of $b_1$ to $x$ as we choose,
    see Figure~\ref{fig:size-reduction}.

\begin{figure}
  \begin{center}
  \begin{tikzpicture}
    \draw[dotted] (-3,0) -- (4,0);
    \draw[dotted] (-3,2.3) -- (4,2.3);
    \draw (0,-1) -- (0,3) node[above] {$\langle b_1 \rangle^\bot$};
    \draw[dashed] (1,-1) -- (1,3);
    \draw[dashed] (-1,-1) -- (-1,3);
    \draw[thick,->] (0,0) -- (2,0) node[below] {$b_1$};
    \draw[thick,->] (0,0) -- (-2,0) node[below] {$-b_1$};
    \draw (0.3,0) arc[start angle=0,end angle=90,radius=0.3cm];

    \foreach \x/\y in {0/2.3,3.2/2.3,-0.8/2.3}
      \fill (\x,\y) circle[radius=2pt];
    \draw (0,2.3) node[below right] {$x'$};
    \draw (3.2,2.3) node[below right] {$x$};
    \draw (-0.8,2.3) node[above left] {$x - 2b_1$};

    \draw[->] (3.2,2.3) .. controls (2.2,2.5) .. (1.2,2.3);
    \draw[->] (1.2,2.3) .. controls (0.2,2.5) .. (-0.8,2.3);
  \end{tikzpicture}
  \end{center}
  \caption{Every projected lattice point has a pre-image whose component parallel to $b_1$
    has length at most $\|b_1\|_2 / 2$.}
  \label{fig:size-reduction}
\end{figure}

    This implies that if there were an $x' \in \pi_1(\Lambda)$
    such that a sequence in $\pi_1(\Lambda)$ has $x'$ as its limit,
    infinitely many pre-images of this sequence would have to lie
    in a ball of radius slightly larger than $\|b_1\|_2/2$
    around a pre-image of $x'$.
    This is impossible by Lemma~\ref{lemma:finitely-many-points-in-bounded-region},
    and therefore $\pi_1(\Lambda)$ is a discrete subset of $\R^d$.
  \end{proof}
  Now, $\pi_1(\Lambda)$ is a lattice of lower dimension,
  so we know that it has a basis $B' = (b_2',\ldots,b_d')$ by the induction hypothesis.
  Let $b_2, \ldots, b_d \in \Lambda$ be arbitrary pre-images of $b_2', \ldots, b_d'$, respectively.

  \begin{claim}
    $b_1, b_2, \ldots, b_d$ is a basis of $\Lambda$.
  \end{claim}
  \begin{proof}
    Clearly, $\Lambda(b_1,b_2,\ldots,b_d) \subseteq \Lambda$.
    For the reverse inclusion, let $x \in \Lambda$.
    By construction, we have
    \[
      \pi_1(x) = a_2 b_2' + \dots + a_d b_d'
    \]
    for some $a_2, \ldots, a_d \in \Z$.
    Let
    \[
      x' := a_2 b_2 + \dots + a_d b_d \in \Lambda
    \]
    be a pre-image of $\pi_1(x)$.
    Note that $x - x' \in \Ker \pi_1 = \R b_1$, that is:
    \[
      x - x' = a_1 b_1
    \]
    Furthermore, $x' - x \in \Lambda$.
    Since we chose $b_1$ to be a primitive vector of the lattice, this implies $a_1 \in \Z$,
    and we have
    \[
      x = a_1 b_1 + a_2 b_2 + \dots + a_d b_d \in \Lambda(b_1,b_2,\ldots,b_d) \qedhere
    \]
  \end{proof}
  This completes the induction step of the proof.
\end{proof}

\begin{remark}
  When we recursively choose $b_1$ as a shortest vector,
  the basis resulting from the construction of the proof is called
  an \emph{HKZ basis}, where the letters refer to Hermite, Korkine, and Zolotareff.
\end{remark}



\section*{Exercises}

\begin{enumerate}
  \item
  \begin{enumerate}[(a)]
    \item
    Let $\Lambda' \subseteq \Lambda$ be full-dimensional lattices and let $B'$ be a basis of $\Lambda'$.
    Recall that $\cP_{B'}$ is the half-open parallelepiped spanned by the columns of $B'$.
    Show:
    \[ \#(\cP_{B'} \cap \Lambda) = \frac{\det \Lambda'}{\det \Lambda} = |\Lambda : \Lambda'| \]

    \item Extend the previous statement to lattices of the same dimension that are not full-dimensional.
  \end{enumerate}
\end{enumerate}

% Copyright 2013 Nicolai Hähnle <nhaehnle@gmail.com>
%
% This work is licensed under the Creative Commons Attribution-ShareAlike 3.0
% Unported License, see http://creativecommons.org/licenses/by-sa/3.0/
%
% Among other things, this means that yes, you may take e.g. illustrations from
% the book and use them in your own work. However, (a) you must give proper
% attribution by naming me as its original author and (b) you must make your
% derivative work available under the same or similar license terms.
%
% See the Creative Commons website for the exact licensing terms.

\chapter{Reduced bases, the LLL algorithm, and approximating shortest vectors}
\label{chapter:basis-reduction-LLL}

In this chapter, we will see the lattice basis reduction algorithm due to
Lenstra, Lenstra, and Lovász~\cite{MR682664}, which we will use to approximate the shortest
vector problem.

Imagine being given a lattice basis $B \in \Q^{d \times d}$.\footnote{We
restrict our attention to rational numbers because it is unclear how to deal with arbitrary real numbers
on a computer. With some care, many but not all statements can be extended to theoretical models of computation
that allow computation with real numbers, but we will not address those questions here.}
The simplest and stupidest thing you could possibly do to approximate a shortest vector in $\Lambda(B)$
is to just return the shortest vector of this basis.
If the basis vectors are orthogonal, you will even find a shortest lattice vector
in this way.
This should be clear intuitively:
\begin{center}
  \begin{tikzpicture}
    \fill (0,0) circle (2pt) node[below left] {$0$};
    \draw[thick,->] (0,0) -- (4,0) node[below] {$b_2$};
    \draw[thick,->] (0,0) -- (0,2) node[left] {$b_1$};
  \end{tikzpicture}
\end{center}
Let us solidify our intuition formally.
For every lattice vector $x \in \Lambda(B)$,
we have
\[
  \|x\|_2^2 = \|a_1b_1 + \dots + a_db_d\|_2^2 = a_1^2 \|b_1\|_2^2 + \dots + a_d^2 \|b_d\|_2^2 \geq \|b_k\|_2^2,
\]
where $a_j \in \Z$ for all $j$,
and the last inequality holds for any $k$ with $a_k \neq 0$.
This gives us a lower bound on the length of any non-zero lattice vector,
and it just so happens that this lower bound coincides with the length of a shortest vector in the given basis.

We can extend at least part of this argument to non-orthogonal bases.
Consider the following situation:
\begin{center}
  \begin{tikzpicture}
    \fill (0,0) circle (2pt) node[below left] {$0$};
    \draw[thick,->] (0,0) -- (4,0) node[below] {$b_1$};
    \draw[thick,->] (0,0) -- (4.4,0.5) node[above] {$b_2$};
    \draw[thick,->] (0,0) -- (0.0,0.5) node[left] {$b_2^\star$};
    \draw[dotted] (0,0.5) -- (4.4,0.5);
  \end{tikzpicture}
\end{center}
Every non-zero lattice vector $x$ lies either on the line $\langle b_1 \rangle$ through $b_1$,
in which case its length is at least $\|b_1\|_2$;
or it lies on a line parallel to $\langle b_1 \rangle$.
However, every such line that contains a lattice point has distance at least $\|b_2^\star\|_2$
from the origin, where $b_2^\star = \pi_1(b_2)$ is the orthogonal projection of $b_2$
onto the orthogonal complement of $\langle b_1 \rangle$.

So $\ell := \min\{\|b_1\|_2, \|b_2^\star\|_2\}$ is a lower bound on the length $\lambda_1$
of a non-zero lattice vector.
This means that we can prove an approximation ratio of $\frac{\|b_1\|_2}{\ell}$
for the simple algorithm that just returns the first lattice basis vector.
Unfortunately, this ratio can be arbitrarily large depending on the given basis.

We will first formalize the preceding ideas using the Gram-Schmidt orthogonalization of a basis,
and will then discuss ways of finding a ``good'' basis in the sense that the resulting approximation ratio
is bounded. As the picture suggests, ``good'' in this context entails ``close to orthogonal''.



\section{Gram-Schmidt orthogonalization}

\begin{definition}
  Let $B = (b_1, \ldots, b_r) \in \R^{d \times r}$ be a lattice basis.
  Its \emph{Gram-Schmidt orthogonalization} $B^\star = (b_1^\star, \ldots, b_r^\star) \in \R^{d \times r}$
  is defined via
  \[
    b_k^\star := \pi_{k-1}(b_k)
      = b_k - \sum_{j=1}^{k-1} \underbrace{\frac{{b_k}^T b_j^\star}{{b_j^\star}^T b_j^\star}}_{=: \mu_{jk}} b_j^\star
  \]
  Recall that $\pi_{k-1} : \R^d \to \langle b_1, \ldots, b_{k-1} \rangle$ is the orthogonal projection
  onto the orthogonal complement of the space spanned by $b_1, \ldots, b_{k-1}$.
\end{definition}
\begin{lemma}
  $B^\star$ is orthogonal, that is, ${b_j^\star}^T b_k^\star = 0$ for all $j \neq k$.
\end{lemma}
\begin{proof}
  Given $j < k$, we simply compute, using the definition of $b_k^\star$:
  \begin{align*}
    {b_j^\star}^T b_k^\star
      &= {b_j^\star}^T b_k - \sum_{i=1}^{k-1} \mu_{ik} \underbrace{{b_j^\star}^T b_i^\star}_{= 0 \text{ for } j \neq i} \\
      &= {b_j^\star}^T b_k - \mu_{jk} {b_j^\star}^T b_j^\star \\
      &= {b_j^\star}^T b_k - \frac{b_k^T b_j^\star}{{b_j^\star}^T b_j^\star} {b_j^\star}^T b_j^\star = 0
  \end{align*}
  To make the second step work, we proceed by induction first over $k$ and then over $j < k$.
\end{proof}

We will often use the definition in its rearranged form:
\[
  b_k = b_k^\star + \sum_{j < k} \mu_{jk} b_j^\star
\]
We often use this set of equations for $k = 1\dots r$ in matrix form:
\[
  B = B^\star M, \text{ where } M = \begin{pmatrix}
                                      1 & \mu_{12} & \dots  & \mu_{1r} \\
                                      0 &    1     &        & \\
                                      \vdots &     & \ddots & \vdots \\
                                      0 &          &        &   1
                                    \end{pmatrix}
\]
Let us formalize the discussion at the beginning of this chapter.
\begin{lemma}
  \label{lemma:svp-lower-bound-gso}
  For all $x \in \Lambda(B) \setminus \{0\}$ we have $\|x\|_2 \geq \min_j \|b_j^\star\|_2$.
\end{lemma}
\begin{proof}
  By definition, we can write $x = a_1 b_1 + \dots + a_r b_r$,
  where $a_j \in \Z$ and there is at least one $a_j \neq 0$.
  In fact, we can write
  \[
    x = a_1 b_1 + \dots + a_k b_k
  \]
  where $k$ is the \emph{last} index with $a_k \neq 0$.
  Now, we compute a change of basis to $B^\star$:
  \begin{align*}
    x &= (a_1 + \mu_{12} a_2 + \dots + \mu_{1k} a_k) b_1^\star \\
      &+ (a_2 + \mu_{23} a_3 + \dots + \mu_{2k} a_k) b_2^\star \\
      &+ \dots \\
      &+ (a_{k-1} + \mu_{k-1,k} a_k) b_{k-1}^\star \\
      &+ a_k b_k^\star
  \end{align*}
  Using Pythagoras' theorem, it follows that
  \begin{align*}
    \|x\|_2 &\geq |a_k| \|b_k^\star\|_2 \geq \|b_k^\star\|_2 \geq \min_j \|b_j^\star\|_2 \qedhere
  \end{align*}
\end{proof}



\section{Reduced bases}

Recall the example basis from the beginning of the chapter,
and recall the observations from the proof of Lemma~\ref{lemma:every-lattice-has-basis}
about lattice points in a certain ``corridor'' around the orthogonal complement of $b_1$.
This immediately suggests that we can get a ``better'', more orthogonal basis
by replacing $b_2$ by $b_2 - b_1$ without changing the lattice:
\begin{center}
  \begin{tikzpicture}
    \fill (0,0) circle (2pt) node[below left] {$0$};
    \draw[thick,->] (0,0) -- (4,0) node[below] {$b_1$};
    \draw[thick,->] (0,0) -- (4.4,0.5) node[above] {$b_2$};
    \draw[thick,->] (0,0) -- (0.4,0.5) node[above] {$b_2 - b_1$};
    \draw[thick,->] (0,0) -- (0.0,0.5) node[left] {$b_2^\star$};
    \draw[dotted] (0,0.5) -- (4.4,0.5);
  \end{tikzpicture}
\end{center}
Visual inspection tells us that $b_2 - b_1$ is a shortest vector.
However, we can get an even more orthogonal basis:
note that this new vector is shorter than $\|b_1\|$,
and if we exchange the two basis vector, the new orthogonalization suggests another ``size reduction'' step:
\begin{center}
  \begin{tikzpicture}
    \fill (0,0) circle (2pt) node[below left] {$0$};
    \draw[thick,->] (0,0) -- (4,0) node[below] {$b_2'$};
    \draw[thick,->] (0,0) -- (0.4,0.5) node[above] {$b_1'$};
    \draw[thick,->] (0,0) -- (2.44,-1.95) node[right] {${b_2'}^\star$};
    \draw[dotted] (2.44,-1.95) -- (4,0);
    \draw[thick,->] (0,0) -- (2.4,-2) node[below] {$b_2' - 4b_1'$};
  \end{tikzpicture}
\end{center}
Surprise: the lattice has an almost exactly orthogonal basis!

The two types of operations we have performed to get to a better basis
suggest a definition of a ``good'' basis as one where these operations can no longer be performed:
\begin{definition}[$d=2$]
  \label{def:reduced-basis-2dim}
  A basis $B \in \R^{d\times 2}$ of a two-dimensional lattice is \emph{reduced} if
  \begin{enumerate}
    \item $|\mu_{12}| \leq 1/2$ (so called \emph{size-reducedness}) and
    \item $\|b_1\|_2 \leq \|b_2\|_2$.
  \end{enumerate}
\end{definition}
Let us contemplate what a reduced basis buys us.
\begin{center}
  \begin{tikzpicture}
    \fill[black!10] (-1.5,4) -- (120:3cm) arc[start angle=120, end angle=60, radius=3cm] -- (1.5,4);
    \draw[dotted] (-4,0) -- (4,0);
    \draw[dotted] (0,-1.5) -- (0,4);
    \fill (0,0) circle (2pt) node[below left] {$0$};
    \draw[thick,->] (0,0) -- (3,0) node[below right] {$b_1$};
    \draw (-30:3cm) arc[start angle=-30,end angle=210,radius=3cm];
    \draw (1.5,-1.5) -- (1.5,4);
    \draw (-1.5,-1.5) -- (-1.5,4);

    \draw[dotted] (-1.4,2.8) -- (0,2.8);
    \draw[thick,->] (0,0) -- (-1.4,2.8) node[above right] {$b_2$};
    \draw[thick,->] (0,0) -- (0,2.8) node[above right] {$b_2^\star$};
  \end{tikzpicture}
\end{center}
The definition of reduced basis implies that $b_2$ must lie in the shaded region
(or its mirror image that points down).
With this picture in mind, it is not difficult to show that $b_1$ is a shortest lattice vector.

For our purposes and with Lemma~\ref{lemma:svp-lower-bound-gso} in mind,
it is interesting to see that $b_2^\star$ can be shorter than $b_1$, but not by much:
\begin{fact}
  \label{fact:reduced-basis-2dim-size-bound}
  In a reduced basis, $\|b_2^\star\|_2 \geq \sqrt{3/4} \|b_1\|_2$.
\end{fact}
\begin{proof}
  Writing $b_2 = b_2^\star + \mu_{12} b_1^\star$, taking squares,
  and remembering that $b_1 = b_1^\star$,
  the second condition of Definition~\ref{def:reduced-basis-2dim} becomes
  \[
    \|b_1^\star\|_2^2 \leq \|b_2^\star + \mu_{12} b_1^\star\|_2^2 = \|b_2^\star\|_2^2 + \mu_{12}^2 \|b_1^\star\|_2^2
  \]
  Rearranging, we get
  \[
    \|b_2^\star\|_2^2 \geq (1 - \mu_{12}^2) \|b_1^\star\|_2^2 \geq \frac{3}{4} \|b_1^\star\|_2^2,
  \]
  where the second inequality follows from the first condition of Definition~\ref{def:reduced-basis-2dim}.
  Now we simply take square roots again to obtain the result.
\end{proof}
The proof of this fact suggests the following non-trivial generalization of
Definition~\ref{def:reduced-basis-2dim} to arbitrary dimension:
\begin{definition}
  A lattice basis is \emph{reduced} if
  \begin{enumerate}
    \item $|\mu_{ij}| \leq 1/2$ for all $i < j$ and
    \item $\|b_j^\star\|_2^2 \leq \|b_{j+1}^\star + \mu_{j,j+1} b_j^\star\|_2^2$ for all $j$ within range.
  \end{enumerate}
\end{definition}

\begin{remark}
  The definitions of reduced bases coincide for $d = 2$.
  Furthermore, the definition can be understood as saying that every ``block''
  of adjacent basis vectors is reduced in the sense that the vectors
  $\pi_{j-1}(b_j), \pi_{j-1}(b_{j+1})$ form a reduced basis of a certain $2$-dimensional lattice.
  This perspective leads to a generalized notion of
  basis reduction that has been studied by Schnorr~\cite{MR918090}.
\end{remark}

\begin{lemma}
  \label{lemma:size-bounds-reduced-basis}
  In a reduced basis $B$ of a lattice of dimension $r$, one has:
  \begin{enumerate}
    \item $\|b_{j+1}^\star\|_2 \geq \sqrt{3/4} \|b_j^\star\|_2$,
    \item $\|b_j^\star\|_2 \geq (3/4)^{(j-1)/2} \|b_1\|_2$, and
    \item $b_1$ is a $(4/3)^{(r-1)/2}$-approximation to a shortest non-zero vector of $\Lambda(B)$.
  \end{enumerate}
\end{lemma}
\begin{proof}
  The first statement is analogous to the proof of Fact~\ref{fact:reduced-basis-2dim-size-bound},
  and the second statement follows from the first by induction on $j$.

  The last statement combines the second statement with Lemma~\ref{lemma:svp-lower-bound-gso}:
  Let $x \in \Lambda(B) \setminus \{ 0 \}$.
  We can compute
  \[
    \|x\|_2 \geq \min_j \|b_j^\star\|_2 \geq \min_j (3/4)^{(j-1)/2} \|b_1\|_2 = (3/4)^{(r-1)/2} \|b_1\|_2.
  \]
  Rearranging, we get that
  \[
    \|b_1\|_2 \leq (4/3)^{(r-1)/2} \|x\|_2
  \]
  for all $x \in \Lambda(B) \setminus \{ 0 \}$.
  In particular, the inequality holds when $x$ is a shortest non-zero vector of $\Lambda(B)$,
  which is exactly the definition of an approximation.
\end{proof}

In the next sections, we will address the question
whether reduced bases always exist and whether we can compute them efficiently.



\section{The determinant of general lattices}
\label{sec:determinant-general-lattices}

Previously, we defined the determinant of a full-dimensional lattice $\Lambda$
as $\det \Lambda = |\det B|$ for some basis $B$ of $\Lambda$,
and we saw that $\det \Lambda = \vol \cP_B$.
We would like a definition of the determinant of a general lattice that coincides
with the definition we already have.
The definitions should coincide not only in the sense that they are equal for full-dimensional lattices,
but also in the sense that we get the same answer for $\Lambda \subset \R^d$
and for $\tau(\Lambda) \subset \R^r$ where $r = \dim\Lambda$ and $\tau$ is an orthogonal transformation
(that is, a transformation that preserves scalar products).

\begin{lemma}
  \label{lemma:determinant-b-star-full-dim}
  Let $B \in \R^{d \times d}$ be a basis of a full-dimensional lattice $\Lambda$.
  Then $\det \Lambda = \prod_{j=1}^d \|b_j^\star\|_2$.
\end{lemma}
\begin{proof}
  Geometrically, this is true because Gram-Schmidt orthogonalization transforms
  the fundamental parallelepiped $\cP_B$ into an \emph{orthogonal} parallelepiped $\cP_{B^\star}$
  with side lengths $\|b_j^\star\|_2$
  by a sequence of shearing operations that leave the volume unchanged.

  More formally, we can write $B = B^\star M$, where $M$ is upper triangular with $1$s on the diagonal.
  Hence $\det B = \det B^\star$. We compute
  \[
    (\det B^\star)^2 = \det {B^\star}^T B^\star = \det \begin{pmatrix}
        \|b_1^\star\|_2^2 &        & 0 \\
                          & \ddots &   \\
        0                 &        & \|b_d^\star\|_2^2
      \end{pmatrix} = \prod_j \|b_j^\star\|_2^2
  \]
  and the claim follows.
\end{proof}

\begin{definition}
  Let $\Lambda \subset \R^d$ be a lattice of dimension $r$.
  Its \emph{determinant} is given by $\det\Lambda := \sqrt{\det(B^TB)}$
  where $B \in \R^{d \times r}$ is a basis of $\Lambda$.
\end{definition}

To see that this definition is well-defined, let $B' \in \R^{d \times r}$ be another basis of $\Lambda$.
By the discussion around Lemma~\ref{lemma:basis-exchange-is-unimodular},
there is a unimodular matrix $U \in \Z^{r \times r}$ such that $B = B'U$.
We can compute
\[
  \det(B^T B) = \det(\underbrace{U^T}_{r \times r} \underbrace{{B'}^T B'}_{r \times r} \underbrace{U}_{r \times r})
    = \underbrace{\det(U)^2}_{=1} \det({B'}^T B') = \det({B'}^T B')
\]
That is, the definition of determinant does not depend on the choice of basis.

\begin{lemma}
  \label{lemma:determinant-b-star}
  Let $B \in \R^{d \times r}$ be a basis of the lattice $\Lambda$.
  Then $\det \Lambda = \prod_{j=1}^r \|b_j^\star\|_2$.
\end{lemma}
\begin{proof}
  The proof is analogous to the proof of Lemma~\ref{lemma:determinant-b-star-full-dim}.
\end{proof}

The norms $\|b_j^\star\|_2$ are invariant under orthogonal transformations,
and so this Lemma is one way to confirm that our original goal has been reached:
the new definition of the determinant of a lattice $\Lambda \subseteq \R^d$ of dimension $r$
is equal to the determinant that we obtain according to the original definition
after embedding $\Lambda$ into $\R^r$ by an orthogonal transformation.




\section{Existence of reduced bases}

Let us consider the following stylized algorithm for basis reduction:
\begin{codebox}
  \Procname{$\proc{Reduce}(B \in \Z^{d \times d})$}
  \li Perform size reduction: ensure $|\mu_{ij}| \leq 1/2$ for all $i < j$ without changing $\Lambda(B)$
  \li \If $\exists j: \|b_j^\star\|_2^2 > \|b_{j+1}^\star + \mu_{j,j+1} b_j^\star\|_2^2$
  \li \Then Swap $b_j \leftrightarrow b_{j+1}$
  \li       \Goto 1.
      \End
  \li \Return $B$
\end{codebox}
It is clear by construction that \emph{if} \proc{Reduce} terminates,
it has computed a reduced basis.
To show \emph{that} \proc{Reduce} terminates,
we analyze the potential function
\begin{align*}
  \Phi(B) &:= \prod_{j=1}^d \prod_{i=1}^j \|b_i^\star\|_2^2 \\
          &= \prod_{j=1}^d (\det \Lambda_j)^2,
\end{align*}
where $\Lambda_j := \Lambda(b_1,\ldots,b_j)$.
\begin{lemma}
  $\Phi(B) \in \N_{\geq 1}$.
\end{lemma}
\begin{proof}
  By definition, $(\det\Lambda_j)^2 = \det(B_j^T B_j)$,
  where $B_j \in \Z^{d \times j}$ contains the first $j$ columns of $B$.
  So $(\det\Lambda_j)^2$ is a positive integer, and the same holds for the product $\Phi(B)$.
\end{proof}

We leave a precise analysis of the size reduction step as an exercise.
For now, it suffices to know that size reduction performs a sequence
of elementary basis transformations of the form: replace $b_j$ by $b_j + \alpha b_i$,
where $i < j$ and $\alpha \in \Z$.\footnote{$\alpha$ is chosen so that $|\mu_{ij}| \leq 1/2$ after the transformation.}
Such transformations do not change the $\Lambda_j$ for any $j$, and therefore:

\begin{lemma}
  $\Phi(B)$ is not changed by size reduction.
\end{lemma}

\begin{lemma}
  \label{lemma:reduce-phi-decreases}
  $\Phi(B)$ strictly decreases every time line~3 of \proc{Reduce} executes.
\end{lemma}
\begin{proof}
  Let us denote by $B$ and $B'$ the lattice bases before and after the swap of $b_j$ and $b_{j+1}$,
  respectively.
  First, note that $\Lambda_i = \Lambda_i'$ for all $i \neq j$, so that
  \[
    \frac{\Phi(B')}{\Phi(B)} = \frac{(\det\Lambda_j')^2}{(\det\Lambda_j)^2}
      = \frac{\prod_{i=1}^j \|{b_i'}^\star\|_2^2}{\prod_{i=1}^j \|{b_i}^\star\|_2^2}
      = \frac{\|{b_j'}^\star\|_2^2}{\|{b_j}^\star\|_2^2}
  \]
  From the definition of Gram-Schmidt orthogonalization,
  it follows that
  \[ {b_j'}^\star = \pi_{j-1}'(b_j') = \pi_{j-1}(b_{j+1}) = b_{j+1}^\star + \mu_{j,j+1} b_j^\star \]
  Therefore,
  \[
    \Phi(B') = \frac{\|b_{j+1}^\star + \mu_{j,j+1} b_j^\star\|_2^2}{\|b_j^\star\|_2^2} \Phi(B) < \Phi(B) \qedhere
  \]
\end{proof}

To summarize: $\Phi(B)$ is integral, positive,
and decreases at least by $1$ in every iteration of \proc{Reduce}.
This means that \proc{Reduce} will eventually terminate.
We performed our analysis on lattices $\Lambda \subseteq \Z^d$,
but by a simple scaling by the lowest common denominator,
it applies equally to any rational lattice. We conclude:

\begin{theorem}
  Every rational lattice has a reduced basis.
\end{theorem}

Unfortunately, the bound on the number of iterations given by Lemma~\ref{lemma:reduce-phi-decreases}
is $\Phi(B)$ for the input lattice basis $B$.
This is a very large number: sufficient for an existence statement,
but nothing to be proud of in terms of algorithm analysis.
We do not know how to improve this analysis.
What we do know is that we can get a significantly better analysis if we slightly relax our notion of reduced basis.


\section{\texorpdfstring{$\delta$}{delta}-reduced bases}

\begin{definition}
  A lattice basis is $\delta$-reduced if
  \begin{enumerate}
    \item $|\mu_{ij}| \leq 1/2$ for all $i < j$ and
    \item $\delta \|b_j^\star\|_2^2 \leq \|b_{j+1}^\star + \mu_{j,j+1} b_j^\star\|_2^2$
  \end{enumerate}
\end{definition}
In the following, we will fix $\delta = 3/4$ for concreteness.
Such a basis is also called LLL-reduced,
after the work of Lenstra, Lenstra, and Lovász~\cite{MR682664}.

Let us modify the \proc{Reduce} accordingly:
\begin{codebox}
  \Procname{$\proc{LLL-Reduce}(B \in \Z^{d \times d})$}
  \li Perform size reduction: ensure $|\mu_{ij}| \leq 1/2$ for all $i < j$ without changing $\Lambda(B)$
  \li \If $\exists j: \frac{3}{4} \|b_j^\star\|_2^2 > \|b_{j+1}^\star + \mu_{j,j+1} b_j^\star\|_2^2$
  \li \Then Swap $b_j \leftrightarrow b_{j+1}$
  \li       \Goto 1.
      \End
  \li \Return $B$
\end{codebox}

We can now give an improved version of Lemma~\ref{lemma:reduce-phi-decreases} for this slightly modified algorithm:

\begin{lemma}
  \label{lemma:lll-reduce-phi-decreases}
  $\Phi(B)$ decreases at least by a factor $3/4$ every time line~3 of \proc{LLL-Reduce} executes.
\end{lemma}
\begin{proof}
  As in the proof of Lemma~\ref{lemma:reduce-phi-decreases},
  we call $B'$ the basis after swapping $b_j$ and $b_{j+1}$ and obtain:
  \[
    \Phi(B') = \frac{\|b_{j+1}^\star + \mu_{j,j+1} b_j^\star\|_2^2}{\|b_j^\star\|_2^2} \Phi(B) < \frac{3}{4} \Phi(B) \qedhere
  \]
\end{proof}

\begin{lemma}
  \label{lemma:lll-iterations}
  The number of iterations of \proc{LLL-Reduce} is bounded by $O(d^2 \log dM)$,
  where $M$ is an upper bound on the absolute value of the entries of the initial input basis $B$.
\end{lemma}
\begin{proof}
  Let $B^{(t)}$ be the basis after $t$ executions of line~3 of \proc{LLL-Reduce}, so that $B^{(0)} = B$.
  Then
  \begin{align}
    \label{eq:LLL-Phi-iterations}
    1 &\leq \Phi(B^{(t)}) < (3/4)^t \Phi(B)
  \end{align}
  Recall that $\Phi(B) = \prod_j \det(B_j^T B_j)$,
  where $B_j = (b_1, \ldots, b_j)$.
  Each component of $B_j^T B_j$ is integer and bounded in absolute value by $dM^2$.
  Using the Hadamard bound, we get:
  \[
    \det(B_j^T B_j) \leq \prod_{i=1}^j \| (B_j^T B_j)_i \|_2 \leq (d^2M^2)^{d/2}
  \]
  This gives us
  \[
    \Phi(B) \leq (d^2 M^2)^{d^2/2}.
  \]
  Plugging this into~\eqref{eq:LLL-Phi-iterations} and taking logarithms, we get
  \[
    t \log(4/3) < d^2 \log(dM) \qedhere
  \]
\end{proof}

\begin{theorem}
  \proc{LLL-Reduce} computes a $3/4$-reduced basis in polynomial time
  in the encoding length of the input.
\end{theorem}
\begin{proof}
  Besides Lemma~\ref{lemma:lll-iterations},
  there are two missing ingredients.
  First, we need a proper analysis of the size reduction step to argue
  that a polynomial number of basic arithmetic operations (such as addition, multiplication, and division) suffices.
  Each of those operations can be implemented in polynomial time in the encoding length of its inputs,
  so the last missing ingredient is a proof that the intermediate numbers in the algorithm
  are bounded in terms of the encoding size of the initial input basis.

  We leave those detailed proofs as an exercise.
\end{proof}

Being able to compute a $3/4$-reduced basis efficiently is all well and good.
It only becomes useful, however, because we an analogue of Lemma~\ref{lemma:size-bounds-reduced-basis} holds.

\begin{lemma}
  \label{lemma:lll-reduced-properties}
  In a $3/4$-reduced basis of a lattice of dimension $r$, one has:
  \begin{enumerate}
    \item $\| b_{j+1}^\star \|_2^2 \geq \frac{1}{2} \|b_j^\star\|_2^2$,
    \item $\| b_j^\star \|_2^2 \geq (1/2)^{j-i} \|b_i\|_2^2$ for all $i < j$, and
    \item $b_1$ is a $2^{(r-1)/2}$-approximation to a shortest non-zero vector of $\Lambda(B)$.
  \end{enumerate}
\end{lemma}
\begin{proof}
  From the definition of $3/4$-reduced basis, we get
  \[
    \frac{3}{4} \|b_j^\star \|_2^2 \leq \| b_{j+1}^\star + \mu_{j,j+1} b_j^\star \|_2^2
      = \| b_{j+1}^\star \|_2^2 + \mu_{j,j+1}^2 \|b_j^\star\|_2^2
  \]
  which we can rearrange as
  \[
    \| b_{j+1}^\star \|_2^2 \geq (3/4 - \underbrace{\mu_{j,j+1}^2}_{\leq 1/4}) \|b_j^\star\|_2^2 \geq \frac{1}{2} \|b_j^\star\|_2^2.
  \]
  From this, the second statement follows by induction,
  and the third statement is a simple application of
  Lemma~\ref{lemma:svp-lower-bound-gso} analogous to the proof of Lemma~\ref{lemma:size-bounds-reduced-basis}.
\end{proof}







\section{Nearest plane approximation for the closest vector problem}
\label{sec:nearest-plane-approximation}

Now that we can approximate the shortest vector problem in the $\ell_2$-norm
in polynomial time to within an exponential factor,
we can ask whether the same is possible for the closest vector problem.
Given a lattice basis $B \in \Q^{d \times d}$ and a target vector $t \in \Q^d$,
can we find a vector $x \in \Lambda(B)$ that minimizes (at least approximately) $\|x-t\|_2$?

Here is a first idea.
Given the target vector $t$, we can compute $\lambda \in \Q^d$
such that $t = B \lambda$.
We could then round each component of $\lambda$ individually
and obtain a lattice vector $B \lceil \lambda \rfloor$.
Clearly, this is going to be a very bad answer if $B$ is a bad basis.
But what if $B$ is LLL-reduced?

It turns out that the answer is positive:
When $B$ is LLL-reduced, this simple procedure is a $2^{O(d)}$-approximation algorithm~\cite{MR856638}.
However, the proof is somewhat involved.
A different, equally simple idea yields a better approximation algorithm with a simpler proof!

\begin{figure}
\begin{center}
  \begin{tikzpicture}
    \clip (-2.1,-1.1) rectangle (5.1,3.1);
    \foreach \alpha in {-1,0,...,5.1}
      \draw ($\alpha*(1.2,0)$) +(0.4,-1.1) -- +(-1.2,3.3);

    \fill (0,0) circle[radius=2pt] node[left] {$0$};

    \draw[thick,->] (0,0) -- (0.4,2.2) node[right] {$b_d$};
    \draw[thick,->] (0,0) -- (1.06,0.39) node[right] {$b_d^\star$};

    \coordinate (t) at (2.7,1.2);
    \coordinate (t') at ($(2.7,1.2) + 0.37*(1.1,0.4)$);
    \draw (t) -- (t');
    \fill (t) circle[radius=2pt] node[left] {$t$};
    \fill (t') circle[radius=2pt] node[right] {$t'$};
  \end{tikzpicture}
\end{center}
  \caption{Approximating the closest vector problem by recursively projecting onto a lattice hyperplane.}
  \label{fig:nearest-plane-approximation}
\end{figure}

Figure~\ref{fig:nearest-plane-approximation}
shows the parallel translates of $\Lambda' := \Lambda(b_1,\ldots,b_{d-1})$
that contain lattice points.
That is, the parallel lines correspond to $\alpha b_n + \Lambda'$ for $\alpha \in \Z$.
The idea is to project the target vector $t$ orthogonally onto the affine subspace
of the nearest such \emph{lattice hyperplane},
and then recursively approximate the closest vector problem in a $(d-1)$-dimensional lattice.
This is formalized in the algorithm \proc{NearestPlane}:
\begin{codebox}
  \Procname{$\proc{NearestPlane}(B \in \Q^{d \times r}, t \in \Q^d \cap \langle B \rangle)$}
  \li \If $r = 0$
  \li \Then \Return $0$
      \End
  \li Find $\lambda \in \Q^r$ such that $t = B \lambda$.
  \li $t' \gets t + (\lceil \lambda_r \rfloor - \lambda_r) b_r^\star$
  \li \Return $\proc{NearestPlane}((b_1,\ldots,b_{r-1}), t' - \lceil \lambda_r \rfloor b_r) + \lceil \lambda_r \rfloor b_r$
\end{codebox}
We immediately see that the returned vector $x \in \Lambda(B)$ satisfies,
for some numbers $\alpha_i$ with $|\alpha_i| \leq 1/2$:
\[
  \| x - t \|_2^2 = \|\alpha_1 b_1^\star + \dots + \alpha_r b_r^\star \|_2^2 \leq \frac{1}{4}(\|b_1^\star\|_2^2 + \dots + \|b_r^\star\|_2^2).
\]
Let $x^\star \in \Lambda(B)$ be a closest lattice vector to $t$.
Suppose that $x^\star$ does \emph{not} lie on the same lattice hyperplane as $t'$.
Then
\[
  \| x^\star - t \|_2 \geq \frac{1}{2} \|b_r^\star\|_2.
\]
On the other hand,
Lemma~\ref{lemma:lll-reduced-properties} says that if $B$ is LLL-reduced, we have
\[
  \|b_i^\star\|_2^2 \leq 2^{r-i} \|b_r^\star\|_2^2
\]
for all $i$.
These inequalities fit together:
\begin{align*}
  \| x - t \|_2^2 &\leq \frac{1}{4} (2^{r-1} + 2^{r-2} + \dots + 2 + 1) \|b_r^\star\|_2^2 \\
    &\leq 2^{r-2} \|b_r^\star\|_2^2 \\
    &\leq 2^r \| x^\star - t \|_2^2
\end{align*}
That is, if $x^\star$ is \emph{not} on the lattice hyperplane we rounded to -- a situation
that intuition tells us should be a bad one --
\proc{NearestPlane} nevertheless computes a $2^{r/2}$-approximate solution to the closest vector problem.

\begin{theorem}
  If $B$ is LLL-reduced, $\proc{NearestPlane}$ is a $2^{r/2}$-approximation algorithm
  for the closest vector problem.
\end{theorem}
\begin{proof}
  We proceed by induction on the dimension $r$ of the lattice.
  For $r = 0,1$, $\proc{NearestPlane}$ does in fact return a closest vector,
  so let us consider the case $r \geq 2$.

  We have seen that we get a $2^{r/2}$-approximation when $x^\star$ lies on a different lattice hyperplane than $t'$,
  so let us now deal with the case when $x^\star$ and $t'$ lie on the same lattice hyperplane.
  Observe that, in this case, $x^\star$ is also a closest vector to $t'$.
  We recursively compute a vector $x' \in \Lambda$ that satisfies
  \[
    \| x' - t' \|_2^2 \leq 2^{r-1} \| x^\star - t' \|_2^2
  \]
  by the induction hypothesis.
  Observe that
  \[
    \underbrace{\frac{ \| x' - t' \|_2^2 }{ \| x^\star - t' \|_2^2 }}_{\geq 1}
    \geq \frac{ \| x' - t' \|_2^2 + \| t' - t \|_2^2 }{ \| x^\star - t' \|_2^2 + \| t' - t \|_2^2 }
    = \frac{ \| x' - t \|_2^2 }{ \| x^\star - t \|_2^2 },
  \]
  where the inequality follows from the fact that a positive fraction approaches $1$
  as the same (positive) quantity is added to both numerator and denominator.
  Read the other way, the inequality yields
  \[
    \frac{ \| x' - t \|_2^2 }{ \| x^\star - t \|_2^2 }
      \leq  \frac{ \| x' - t' \|_2^2 }{ \| x^\star - t' \|_2^2 }
      \leq 2^{r-1}
    \qedhere
  \]
\end{proof}







\section*{Exercises}

\begin{enumerate}
  \item
  \begin{enumerate}[(a)]
    \item Let $b_1,b_2 \in \R^2$ be a reduced basis of $\Lambda$.
      Show: $b_1$ is a shortest vector of $\Lambda$.

    \item Show: For $d=2$,
      the algorithm \proc{Reduce} finds a reduced basis in polynomial time
      in the encoding length of the input.
  \end{enumerate}

  \item
  \begin{enumerate}[(a)]
    \item Let $\Lambda \subseteq \R^d$ be a lattice and
    let $U \subseteq \R^d$ be a subspace spanned by lattice vectors.
    Show: $\Lambda \cap U$ is a lattice, $\dim \Lambda \cap U = \dim U$.

    \item Let $\pi_{U^\bot} : \R^d \to U^\bot$ be the orthogonal projection onto the orthogonal
    complement of $U$.
    Show: $\pi_{U^\bot} (\Lambda)$ is a lattice, $\dim \pi_{U^\bot} (\Lambda) = \dim \Lambda - \dim U$.

    \item Show: $\det \Lambda = \det \Lambda \cap U \cdot \det \pi_{U^\bot} (\Lambda)$
  \end{enumerate}

  \item
    \begin{enumerate}[(a)]
      \item Show that the size reduction step of the LLL algorithm can be performed using $\poly(d)$ arithmetic operations.

        Note: You will probably want two nested loops iterating over the $\mu_{ij}$.
        Carefully evaluate the order in which you process the $\mu_{ij}$ in those loops!

      \item Show that the binary encoding sizes of all intermediate numbers used in the LLL algorithm
        are bounded by $\poly(b)$, i.e. a polynomial in the binary encoding size of the input.
    \end{enumerate}

  \item
    Let $B \in \Q^{d \times d}$ be an LLL-reduced basis of $\Lambda$.
    \begin{enumerate}[(a)]
      \item Let $x = a_1 b_1 + \dots + a_d b_d$ be a shortest vector of $\Lambda$.
        Show: $|a_j| \leq 2^{O(d)}$ for every $j = 1 \dots d$.

        Hint: You may use reverse induction where you start by showing the claim for $j = d$.

      \item Show that a shortest vector of $\Lambda$ can be computed in time $2^{O(d^2)}$.
    \end{enumerate}

  \item
    Let $B \in \Q^{d \times d}$ be an LLL-reduced basis of $\Lambda$ and let $t \in \R^d$.
    \begin{enumerate}[(a)]
      \item Let $x^\star \in \Lambda$ be a closest vector to $t$ and consider the lattice hyperplanes
        orthogonal to $b_d^\star$ (i.e., parallel to $\Lambda' := \Lambda(b_1,\ldots,b_{d-1})$).
        Let $\lambda \in \R^d$ such that $t = B \lambda$
        and let $\alpha \in \Z^d$ such that $x^\star = B \alpha$.

        Show that $|\alpha_d - \lambda_d| \leq 2^{d-2}$.

      \item Show that a closest vector to $t$ in $\Lambda$ can be computed in time $2^{O(d^2)}$.
    \end{enumerate}

  \item
    Show that given an LLL-reduced basis $B \in \Q^{d \times d}$ and target vector $t \in \Q^d$,
    the following simple rounding procedure gives a $2^{O(d)}$-approximation:
    \begin{codebox}
      \li Compute $\lambda$ such that $t = B \lambda$
      \li \Return $B \lceil \lambda \rfloor$, where $\lceil \lambda \rfloor$ is a nearest integer point.
    \end{codebox}

    Hint: \cite{MR856638}
\end{enumerate}

% Copyright 2013 Nicolai Hähnle <nhaehnle@gmail.com>
%
% This work is licensed under the Creative Commons Attribution-ShareAlike 3.0
% Unported License, see http://creativecommons.org/licenses/by-sa/3.0/
%
% Among other things, this means that yes, you may take e.g. illustrations from
% the book and use them in your own work. However, (a) you must give proper
% attribution by naming me as its original author and (b) you must make your
% derivative work available under the same or similar license terms.
%
% See the Creative Commons website for the exact licensing terms.

\chapter{The Voronoi Cell of a Lattice and its Applications}
\label{chapter:voronoi-cell}

The Voronoi diagram of a point set is defined
to provide the answer to the closest vector problem:
given a point set $P$,
the Voronoi cell of $x \in P$ is the set of points
that are closer to $x$ than to any other point in $P$.

In this chapter, we will see how the Voronoi diagram of a lattice
can be used to solve both the shortest and the closest vector problem
in single exponential time by a deterministic algorithm.
Considering how early Voronoi diagrams were first studied,
and how obvious their usefulness seems in hindsight,
it took a very long to find this algorithm
which is due to Micciancio and Voulgaris~\cite{MR2743283}.


\section{The Voronoi cell of a lattice}

\begin{definition}
  Let $\Lambda \subset \R^d$ be a lattice.
  Let $x \in \Lambda \setminus \{ 0 \}$.
  We define the open half-space
  \[
    H_x := \{ p \in \R^d ~:~ \|p\|_2 < \|x-p\|_2 \}
  \]
  and the (open) \emph{Voronoi cell} of $\Lambda$
  \[
    \cV_\Lambda := \bigcup_{x \in \Lambda\setminus\{0\}} H_x
  \]
  Figure~\ref{fig:voronoi-diagram} shows a $2$-dimensional example.
\end{definition}

\begin{figure}
  \begin{center}
    \begin{tikzpicture}
      \clip (-2.2,-2.3) rectangle (2.2,4.3);

      \fill[black!10]
        (0,1.0625) -- (0.5,0.9375) -- (0.5,-0.9375) -- (0,-1.0625) --
        (-0.5,-0.9375) -- (-0.5,0.9375) --cycle;

      \foreach \y in {-1,0,1,2}
        \foreach \x in {-3,-2,-1,0,1,2}
          \fill ($\x*(1,0) + \y*(0.5,2)$) circle[radius=2pt];

      \draw (0,0) node[below] {$0$};

      \foreach \y in {-1,0,1,2}
        \foreach \x in {-4,-3,-2,-1,0,1,2}
          \draw ($\x*(1,0) + \y*(0.5,2)$) +(0,1.0625) -- +(0.5,0.9375) -- +(0.5,-0.9375) -- +(0,-1.0625);
    \end{tikzpicture}
  \end{center}
  \caption{The Voronoi diagram of a lattice. The Voronoi cell of $0$ is shaded.}
  \label{fig:voronoi-diagram}
\end{figure}

\begin{lemma}
  If $\Lambda$ is full-dimensional, $\overline{\cV_\Lambda}$ is a polytope.

  For a general lattice $\Lambda$,
  $\overline{\cV_\Lambda} = P + \langle \Lambda \rangle^\bot$
  where $P = \overline{\cV_\Lambda} \cap \langle \Lambda \rangle$ is a polytope.
\end{lemma}
\begin{proof}
  For linearly independent vectors $x_1, \ldots, x_d \in \Lambda$,
  we have
  \[
    \cV_\Lambda \subseteq
      \underbrace{H_{x_1} \cap \dots \cap H_{x_d} \cap H_{-x_1} \cap \dots \cap H_{-x_d}}_{=: Q}.
  \]
  The right hand side $Q$ is an open parallelepiped.
  In particular, it is bounded, so $\cV_\Lambda$ is bounded.

  Let $R > 0$ such that $Q \subseteq B(0,R)$.
  Every facet of $\overline{Q}$ lies on a hyperplane of the form
  \[
    \|p\|_2 = \|x - p\|_2 \iff 2p^Tx = x^Tx
  \]
  for some $x = \pm x_j$.
  Note that $\|p\|_2 \geq \frac{1}{2} \|x\|_2$ for all points $p$ on the facet.
  On the other hand, $\|p\|_2 \leq R$ by definition of $R$,
  so that we get $\|\pm x_j\|_2 \leq 2R$ for all $j$.

  We claim that
  \[
    \cV_\Lambda = \bigcap_{x \in \Lambda\setminus\{0\} ~:~ \|x\|_2 \leq 2R} H_x
  \]
  follows.
  The inclusion from left to right is clear from the definition.
  For the inclusion from right to left,
  let $p \in H_x$ for all $x \in \Lambda\setminus\{0\}$ with $\|x\|_2 \leq 2R$.
  In particular, this implies $x \in \overline{Q}$ and therefore $\|x\|_2 \leq R$.
  This in turn implies $x \in H_y$ for all $y$ with $\|y\|_2 > 2R$.
  To summarize,
  \[
     x \in \left( \bigcap_{x \in \Lambda\setminus\{0\} ~:~ \|x\|_2 \leq 2R} H_x \right)
      \cap \left( \bigcap_{y \in \Lambda\setminus\{0\} ~:~ \|y\|_2 > 2R} H_y \right) = \cV_\Lambda
  \]
  which implies the claim.
  That is, $\cV_\Lambda$ is bounded and defined as the intersection of finitely many half-spaces,
  i.e., it is a polytope.

  The statement of the Lemma for general lattices follows
  directly from the observation that $x \in \Lambda$ is a closest lattice vector to $p$
  if and only if it is a closest lattice vector of the orthogonal projection $p'$ of $p$
  onto $\langle \Lambda \rangle$.
\end{proof}

Now that we have established that $\cV_\Lambda$ is a polytope,
a natural algorithmic representation suggests itself:

\begin{definition}
  We say that $x \in \Lambda$ is \emph{Voronoi relevant} if
  \[
    \| p \|_2 \leq \| x - p \|_2 \iff 2p^Tx \leq x^T x
  \]
  is a facet-defining inequality of $\overline{\cV_\Lambda}$.
\end{definition}

Given a representation of $\cV_\Lambda$ in terms of Voronoi relevant vectors,
simple tasks such as testing whether a point is contained in $\cV_\Lambda$
can be solved in time $N \cdot \poly(d, b)$
where $b$ is the encoding size of the coordinates
and $N$ is the number of Voronoi relevant vectors.
Our first goal in this chapter will be to bound this number $N$.
Before we can do this -- and simultaneously show how those vectors
can be found -- we need some basic tools.

\begin{lemma}
  \label{lemma:voronoi-diagram-symmetry}
  Let $x \in \frac{1}{2} \Lambda$.
  The Voronoi diagram of $\Lambda$ is symmetric with respect to $x$.
\end{lemma}
\begin{proof}
  Since the symmetry $x + u \mapsto x - u$ preserves distances,
  we only need to show that the lattice $\Lambda$ is symmetric with respect to $x$.
  Whenever $x + u \in \Lambda$, we have
  \[
    x - u = (x + u) - 2u,
  \]
  where $u \in \frac{1}{2} \Lambda$.
  That is, $x - u \in \Lambda$ as well, completing the proof.
\end{proof}

\begin{lemma}
  \label{lemma:voronoi-relevant-facet-interior}
  Let $x \in \Lambda$ be a Voronoi relevant vector.
  Then $x/2$ lies in the relative interior of the associated facet of $\cV_\Lambda$.
\end{lemma}
\begin{proof}
  The facet associated to $x$ is
  \[
    F = \cV_\Lambda \cap \{ 2p^Tx = x^Tx \}
  \]
  The point $x/2$ lies on the hyperplane defining $F$,
  and by Lemma~\ref{lemma:voronoi-diagram-symmetry},
  the entire Voronoi diagram is symmetric with respect to $x/2$.
  In particular, $F$ is symmetric with respect to $x/2$.
  Since $F$ is convex, this implies the claim.
\end{proof}

\begin{lemma}
  \label{lemma:voronoi-relevant-as-closest}
  Let $x \in \Lambda$.
  Then $x$ is Voronoi relevant if and only if
  $0$ and $x$ are the unique closest vectors to $\frac{1}{2} x$.
\end{lemma}
\begin{proof}
  Let $x$ be Voronoi relevant and
  assume that there is a third vector $y \in \Lambda$ that is equally close or closer.
  \begin{center}
    \begin{tikzpicture}
      \fill (0,0) circle[radius=2pt] node[below left] {$0$};
      \fill (3,0) circle[radius=2pt] node[below right] {$x$};
      \draw[dotted] (-0.5,0) -- (3.5,0);
      \draw[ultra thick] (1.5,-0.4) -- (1.5,0.4) node[right] {$F$};
      \draw (1.5,0) circle[radius=1.5cm];

      \fill (1.5,0) +(30:1.5) circle[radius=2pt] node[right] {$y$};
      \draw[thick] (1.5,0) +(0.188,-0.7) -- +(-0.188,0.7);
      \draw[dotted] (2.8,0.75) -- (0,0);
    \end{tikzpicture}
  \end{center}
  In this case, $x/2$ either lies on the hyperplane bounding $H_y$
  or is cut off by that hyperplane entirely.
  In any case, we have a contradiction with Lemma~\ref{lemma:voronoi-relevant-facet-interior}.

  Now suppose $0$ and $x$ are the unique closest vectors to $\frac{1}{2} x$.
  By compactness, there is a neighborhood of $\frac{1}{2} x$
  in the bounding hyperplane $\partial H_x$
  for which $0$ and $x$ are the unique closest vectors as well.
  The neighborhood is therefore contained in $\overline{\cV_\Lambda}$,
  which means that there is a facet of $\overline{\cV_\Lambda}$ in $\partial H_x$,
  i.e., $x$ is Voronoi relevant.
\end{proof}



\section{Computing Voronoi relevant vectors}

A different perspective on the statement of Lemma~\ref{lemma:voronoi-relevant-as-closest}
is the following.
The point $x/2$ lies in the refined lattice $\frac{1}{2} \Lambda$,
to which we can associate a natural \emph{parity function},
see Figure~\ref{fig:lattice-refinement-parity} for an illustration:
\begin{align*}
  \sigma : \frac{1}{2} \Lambda &\to G := (\frac{1}{2} \Lambda) / \Lambda \cong (\Z_2)^d \\
                   u &\mapsto u + \Lambda
\end{align*}%
\begin{figure}
  \begin{center}
  \begin{tikzpicture}
    \clip (-1,-1) rectangle (4,3);
    \foreach \p/\color in {{(0,0)}/black,{(0.75,0.15)}/red,{(-0.05,0.85)}/green,{(0.7,1)}/blue}
      \foreach \x in {-1,0,1,2}
        \foreach \y in {-2,-1,0,1,2}
          \path[fill=\color] ($\x*(1.5,0.3) + \y*(-0.1,1.7)$) +\p circle[radius=2pt];
  \end{tikzpicture}
  \end{center}
  \caption{A lattice $\Lambda$ in black and its refinement $\frac{1}{2} \Lambda$
    with parities indicated using different colors.}
  \label{fig:lattice-refinement-parity}
\end{figure}%
The Lemma then says that for every Voronoi relevant vector $x \in \Lambda$,
there is a vector $u \in \frac{1}{2} \Lambda$ of non-zero parity
such that $0$ and $x$ are the unique closest vectors to $u$ in $\Lambda$.
For all vectors $u' \in \frac{1}{2} \Lambda$ with the \emph{same} parity $\sigma(u') = \sigma(u)$,
the relative location of closest vectors in $\Lambda$ is the same.
This suggests the following algorithm for finding Voronoi relevant vectors:
\begin{codebox}
  \Procname{$\proc{VoronoiCell}(\Lambda)$}
  \li $X \gets \emptyset$
  \li \For $U \gets \left((\frac{1}{2} \Lambda) / \Lambda\right) \setminus \{ 0 \}$
  \li \Do $u \gets$ representative of $U$
  \li     $x \gets \proc{CVP}(\Lambda, u)$
  \li     $X \gets X \cup \{ 2(x - u), 2(u - x) \}$
      \End
  \li \Return $X$
\end{codebox}
We are purposefully imprecise about how the lattice $\Lambda$ is represented
and how the closest vector problem is to be solved.
For now, the key point is that we can find all Voronoi relevant vectors
by solving $2^d - 1$ closest vector problems:

\begin{lemma}
  \label{lemma:voronoi-cell-computation}
  The set $X$ returned by \proc{VoronoiCell} contains all Voronoi relevant vectors.
\end{lemma}
\begin{proof}
  Let $y \in \Lambda$ be a Voronoi relevant vector.
  By Lemma~\ref{lemma:voronoi-relevant-as-closest},
  the vector $y/2 \in \frac{1}{2}\Lambda \setminus \Lambda$ has exactly two closest vectors,
  $0$ and $y$.

  There is one iteration of $\proc{VoronoiCell}$ in which $\sigma(u) = \sigma(y/2)$
  or, equivalently, $u \equiv y/2 \pmod{\Lambda}$.
  In this iteration, the closest vector subroutine must return
  either $x = u - y/2$ or $x = u + y/2$, see Figure~\ref{fig:voronoi-cell-computation}.
  In both cases, the algorithm adds $y$ and $-y$ to the set $X$.
\end{proof}
\begin{figure}
  \begin{center}
  \begin{tikzpicture}
    \fill (0,0) circle[radius=2pt] node[below left] {$0$};
    \fill (2,0) circle[radius=2pt] node[below right] {$y$};
    \fill (1,0) circle[radius=2pt];
    \draw (1,0) circle[radius=1cm];
    \draw (0,0) -- (2,0);

    \fill (3,2) circle[radius=2pt] node[below] {$u$};
    \fill (4,2) circle[radius=2pt] node[below right] {$u + y/2$};
    \fill (2,2) circle[radius=2pt] node[above left] {$u - y/2$};
    \draw (3,2) circle[radius=1cm];
    \draw (2,2) -- (4,2);

    \draw[dotted] (0,0) -- (2,2);
    \draw[dotted] (1,0) -- (3,2);
    \draw[dotted] (2,0) -- (4,2);
  \end{tikzpicture}
  \end{center}
  \caption{Since $u \equiv y/2 \pmod\Lambda$, the set of closest vectors is identical after translation.}
  \label{fig:voronoi-cell-computation}
\end{figure}

\begin{example}
  The computed set $X$ may contain vectors that are not Voronoi relevant.
  Consider the case $\Lambda = \Z^2$ as shown below with its Voronoi cell.
  \begin{center}
    \begin{tikzpicture}
      \draw[fill=black!10] (-1,1) rectangle (1,-1) node[right] {$\overline{\cV_\Lambda}$};

      \foreach \x in {-1,0,1}
        \foreach \y in {-1,0,1}
          \fill ($\x*(2,0) + \y*(0,2)$) circle[radius=2pt];

      \draw (0,0) node[below] {$0$};

      \draw[fill=white] (1,1) circle[radius=2pt] node[right] {$u$};

      \draw (1,1) circle[radius=1.41cm];
    \end{tikzpicture}
  \end{center}
  During one iteration of the algorithm, $u$ will be as shown (or equivalent modulo $\Lambda$).
  There are four closest vectors to $u$, and no matter which of them is returned,
  the algorithm adds two vectors to $X$ that are not Voronoi relevant because
  the corresponding hyperplane $\partial H_x$ only induces
  a lower-dimensional face of $\overline{\cV_\Lambda}$.
\end{example}

If desired, we can detect vectors in $X$ that are not Voronoi relevant
using $O(N^2 d) = 2^{O(d)}$ arithmetic operations:
according to Lemma~\ref{lemma:voronoi-relevant-facet-interior},
it suffices to check every $x/2$ for $x \in X$ against every other $y \in X$
and compare distances.


\begin{corollary}
  \label{cor:number-of-voronoi-relevant}
  There are at most $2 \cdot (2^d - 1)$ Voronoi relevant vectors.
\end{corollary}



\section{Shortest and closest vectors via the Voronoi cell}

Now that we know how to compute the Voronoi cell,
let us see how it can be used.
First, we see that it contains a list of all shortest vectors of the lattice.

\begin{lemma}
  Let $x \in \Lambda$ be a shortest non-zero vector.
  Then $x$ is Voronoi relevant.
\end{lemma}
\begin{proof}
  Consider the ball of radius $\|x/2\|_2$ around $x/2$:
  \begin{center}
  \begin{tikzpicture}
    \fill (0,0) circle[radius=2pt] node[below left]{$0$};
    \fill (2,0) circle[radius=2pt] node[below right]{$x$};
    \fill (1,0) circle[radius=2pt] node[below]{$x/2$};

    \draw (1,0) circle[radius=1cm];
    \draw (0,0) +(-75:2cm) arc[start angle=-75,end angle=75,radius=2cm];
  \end{tikzpicture}
  \end{center}
  Since $x$ is a shortest vector, $0$ and $x$ are the unique closest vectors to $x/2$.
  By Lemma~\ref{lemma:voronoi-relevant-as-closest}, $x$ is Voronoi relevant.
\end{proof}

\begin{corollary}
  Every lattice has at most $2 \cdot (2^d - 1)$ shortest vectors.
\end{corollary}

\begin{remark}
  Let us place spheres centered at every lattice point of a radius
  such that two spheres may touch but not overlap.
  The number of spheres that touch a fixed sphere is called the \emph{kissing number}
  of the sphere packing,
  and it is equal to the number of shortest vectors in the lattice.
  Hence, the previous Corollary is an upper bound on the kissing number
  of a lattice sphere packing.

  This bound happens to be tight for $d = 2$,
  but considerably better bounds are known in higher dimension.
  A classical source on the study of sphere spackings is~\cite{MR1662447}.
\end{remark}

We can also use the Voronoi cell to compute closest vectors.
This may seem absurd at first,
because we needed to compute closest vectors in the first place to be able to
find the Voronoi relevant vectors.
However, if we need to solve \emph{many} closest vector problems,
then it may be useful to compute the Voronoi cell as a pre-processing step
to speed up future CVP computations.
We will apply this idea in Section~\ref{sec:voronoi-full-algorithm}.

For now, suppose we know the Voronoi cell $\cV$ of a lattice $\Lambda$
and are given a target vector $t \in \R^d$.
\begin{figure}
\begin{center}
  \begin{tikzpicture}
    \clip (-2.2,-2.3) rectangle (2.2,4.3);

    \foreach \y in {-1,0,1,2}
      \foreach \x in {-3,-2,-1,0,1,2}
        \fill ($\x*(1,0) + \y*(0.5,2)$) circle[radius=2pt];

    \draw (0,0) node[below] {$0$};

    \foreach \y in {-1,0,1,2}
      \foreach \x in {-4,-3,-2,-1,0,1,2}
        \draw ($\x*(1,0) + \y*(0.5,2)$) +(0,1.0625) -- +(0.5,0.9375) -- +(0.5,-0.9375) -- +(0,-1.0625);

    \coordinate (t) at (-1.1,3.7);
    \fill (t) circle[radius=2pt] node[left] {$t$};
    \draw (0,0) -- (t);
  \end{tikzpicture}
\end{center}
\caption{We find the Voronoi cell containing $t$ by tracing along the segment $[0,t]$.}
\label{fig:cvp-via-voronoi-cell-tracing}
\end{figure}
The problem of finding a closest lattice vector to $t$
is essentially the problem of finding the Voronoi cell translate that contains $t$.
We will solve that problem by following the line segment $[0,t]$,
see Figure~\ref{fig:cvp-via-voronoi-cell-tracing}.
\begin{codebox}
  \Procname{$\proc{CVP-via-Voronoi-Simple}(\Lambda, \cV_\Lambda, t)$}
  \zi $\cV_\Lambda$ is given by a list that contains all Voronoi relevant vectors
  \li $x \gets 0$
  \li \While $t \not\in x + \overline{\cV_\Lambda}$
  \li \Do Find the point $p$ where $[0,t]$ leaves $x + \cV_\Lambda$
  \li     Let $x' + \cV_\Lambda$, $x' \in \Lambda$, be the Voronoi cell entered at $p$
  \li     $x \gets x'$
      \End
  \li \Return $x$
\end{codebox}
Clearly, the algorithm returns a correct result when it terminates.
Each of the individual steps in this algorithm can be implemented in time $O(N \cdot \poly(d,b))$,
where $N$ is the size of the given list containing the Voronoi relevant vectors.
The test whether $t \in x + \overline{\cV_\Lambda}$ can be done by evaluating the $N$ linear functions
associated with the facets of $\overline{\cV_\Lambda}$.
Similarly, the point $p$ and the facet it lies on can be found by intersecting the line through $0$ and $t$
with each of the hyperplanes defining the facets of $x + \overline{\cV_\Lambda}$.

There is a subtle degenerate situation when the segment $[0,t]$ intersects a lower-dimensional
face of a cell $x + \overline{\cV_\Lambda}$.
This degeneracy can be avoided by an appropriate perturbation of the target vector $t$.

For now, let us simply assume that this degenerate situation does not happen
and continue bounding the running time of the algorithm.
Since the Voronoi cells are convex, every cell is entered at most once.
Since we only enter cells that intersect the segment $[0,t]$,
it suffices to bound the number of such cells
to bound the overall running time of the algorithm.

For every $x + \cV_\Lambda$ encountered in the algorithm,
we have $x \in [0,t] + \cV_\Lambda$
and therefore
\[
  x + \cV_\Lambda \subseteq [0,t] + 2 \cV_\Lambda
\]
Since the open cells are disjoint,
it suffices to bound the volume of $[0,t] + 2 \cV_\Lambda$.
\begin{lemma}
  Let $\lambda \in \R$ such that $t \in \lambda \cV_\Lambda$.
  Then \proc{CVP-via-Voronoi-Simple} visits at most $(\lambda+2)^d$ Voronoi cells of the lattice.
\end{lemma}
\begin{proof}
  By convexity, we have $[0,t] \subseteq \lambda \cV_\Lambda$,
  so that
  \[
    x + \cV_\Lambda \subseteq (\lambda + 2) \cV_\Lambda
  \]
  for every Voronoi cell visited by the algorithm.
  Hence, the number of such cells is bounded by
  \[
    \frac{\vol((\lambda + 2) \cV_\Lambda)}{\vol(x + \cV_\Lambda)}
      = \frac{(\lambda + 2)^d \vol \cV_\Lambda }{\vol \cV_\Lambda}
      = (\lambda + 2)^d \qedhere
  \]
\end{proof}
This seems to be a rather crude bound.
Intuitively, we expect a bound that is linear in the norm of $t$ for very large $t$.
But note that even such a bound -- and it is clear that we cannot possibly do better --
is exponential in the encoding size of $t$, which leads to an unacceptable running time.
The solution lies in geometric scaling, using two simple observations.
\begin{corollary}
  If $t \in 2\cV_\Lambda$,
  \proc{CVP-via-Voronoi-Simple} can be used to solve CVP in time $2^{O(d)} \poly(b)$.
\end{corollary}

The second observation is that the Voronoi cell of $k \Lambda$ is simply $k \cV_\Lambda$.
This means that if $t \not\in 2\cV_\Lambda$,
we can instead search for a closest vector $x'$ to $t$
in a sparser lattice $2^\kappa \cdot \Lambda$
that satisfies $t \in 2 \cdot 2^\kappa \cV_\Lambda$.
Once such a vector $x' \in \Lambda$ is found,
we can shift the problem by $-x'$,
thereby moving $t$ closer to the origin.
This is illustrated in Figure~\ref{fig:cvp-via-scaling}.

\begin{codebox}
  \Procname{$\proc{CVP-via-Voronoi-Scaling}(\Lambda, \cV_\Lambda, t)$}
  \li \If $t \in \cV_\Lambda$
  \li \Then \Return $0$
      \End
  \li $x' \gets \proc{CVP-via-Voronoi-Scaling}(2\Lambda, 2\cV_\Lambda, t)$
  \li $x'' \gets \proc{CVP-via-Voronoi-Simple}(\Lambda, \cV_\Lambda, t - x')$
  \li \Return $x' + x''$
\end{codebox}
Let $\kappa \in \N_0$ be minimal with $t \in 2^\kappa \cV_\Lambda$.
Then it is easy to see by induction over $\kappa$
that the recursion depth given input $t$ is exactly $\kappa$.

We see the correctness of the algorithm via induction over $\kappa$ as well.
It is self-evident if $\kappa = 0$, i.e., $t \in \cV_\Lambda$.
Suppose $\kappa \in \N_0$ is minimal with $t \in \cV_\Lambda$ and $\kappa \geq 1$.
Then we find, by the induction hypothesis,
$x' \in 2\Lambda$ such that $t \in x' + 2\cV_\Lambda$.
We also find $x'' \in \Lambda$ with $(t-x') \in x'' + \cV_\Lambda$,
and therefore $t \in (x' + x'') + \cV_\Lambda$,
which means the returned lattice vector is a closest vector to $t$.

Since we have $(t - x') \in 2\cV_\Lambda$,
each call of $\proc{CVP-via-Voronoi-Simple}$
takes time $2^{O(d)} \poly(b)$.
That is, the overall running of the algorithm is essentially $\kappa \cdot 2^{O(d)} \poly(b)$.
The recursion depth $\kappa$ is bounded by $\log \|t\|_2$ if $\Lambda \subseteq \Z^d$.
That is, after an appropriate scaling, we get:

\begin{theorem}
  Given a list of Voronoi relevant vectors,
  $\proc{CVP-via-Voronoi-Scaling}$ computes a closest vector to $t$ in time $2^{O(d)} \poly(b)$.
\end{theorem}


\begin{figure}
\begin{center}
  \begin{tikzpicture}
    \clip (-2.2,-2.3) rectangle (2.2,4.3);

    \foreach \y in {-1,0,1,2}
      \foreach \x in {-3,-2,-1,0,1,2}
        \fill[black!40] ($\x*(1,0) + \y*(0.5,2)$) circle[radius=2pt];

    \draw (0,0) node[below] {$0$};

    \foreach \y in {-1,0,1,2}
      \foreach \x in {-4,-3,-2,-1,0,1,2}
        \draw[black!40] ($\x*(1,0) + \y*(0.5,2)$) +(0,1.0625) -- +(0.5,0.9375) -- +(0.5,-0.9375) -- +(0,-1.0625);

    \coordinate (t) at (-1.1,3.7);
    \fill (t) circle[radius=2pt] node[left] {$t$};
    \draw (0,0) -- (t);
    \draw (-1,4) node[right] {$x'$};

    \foreach \y in {0,2}
      \foreach \x in {-2,0,2}
        \fill ($\x*(1,0) + \y*(0.5,2)$) circle[radius=2pt];

    \foreach \y in {-2,0,2}
      \foreach \x in {-4,-2,0,2}
        \draw ($\x*(1,0) + \y*(0.5,2)$) +(0,2.125) -- +(1.0,1.875) -- +(1.0,-1.875) -- +(0,-2.125);

  \end{tikzpicture}
\end{center}
  \caption{Solving the closest vector problem via \proc{CVP-via-Voronoi-Scaling}.}
  \label{fig:cvp-via-scaling}
\end{figure}



\section{A single exponential time algorithm for computing the Voronoi cell}
\label{sec:voronoi-full-algorithm}

We find ourselves in an oddly circular situation.
Given some way of computing closest vectors in $\Lambda$,
we can also compute its Voronoi relevant vectors,
and given a way of computing its Voronoi relevant vectors,
we can also computed closest vectors in $\Lambda$.
\begin{center}
  \begin{tikzpicture}
    \node[left,inner sep=3mm,draw] (cvp) at (0,0) {Closest vectors};
    \node[right,inner sep=3mm,draw] (vc) at (3,0) {Voronoi cell};

    \draw[->] ($(cvp.east) + (0,-0.1)$) -- ($(vc.west) + (0,-0.1)$);
    \draw[->] ($(vc.west) + (0,0.1)$) -- ($(cvp.east) + (0,0.1)$);
  \end{tikzpicture}
\end{center}
We will break out of this cycle by a strengthening of the nearest plane method
of Section~\ref{sec:nearest-plane-approximation}.
Given an LLL-reduced basis $B$ of $\Lambda$,
we will see how a closest vector problem in $\Lambda$
can be solved by $2^{d/2}$ closest vector problems in $\Lambda' = \Lambda(b_1,\ldots,b_{d-1})$.

Each of those closest vector problems can be solved using the Voronoi cell of $\Lambda'$,
which has to be computed only once.
The circular dependency above becomes a linear chain of dependencies
in which we compute Voronoi cells of lattices of successively higher dimension,
see Figure~\ref{fig:voronoi-cell-dimension-reduction}.

The key point of the construction is that
even though up to $2^{O(d)}$ closest vector problems are solved in each dimension,
only one Voronoi cell needs to be computed in each dimension.
Hence the overall running time remains singly exponential.
\begin{figure}
  \begin{center}
    \begin{tikzpicture}
      \node[left,inner sep=3mm,draw] (cvp1) at (0,0) {Closest vectors in $\dim = 1$};
      \node[left,inner sep=3mm,draw] (cvp2a) at (0,-1.2) {Closest vectors in $\dim = 2$};
      \node[right,inner sep=3mm,draw] (vc2) at (2.5,-1.8) {Voronoi cell in $\dim = 2$};
      \node[left,inner sep=3mm,draw] (cvp2b) at (0,-2.4) {Closest vectors in $\dim = 2$};
      \node[left,inner sep=3mm,draw] (cvp3a) at (0,-3.6) {Closest vectors in $\dim = 3$};
      \node[right,inner sep=3mm,draw] (vc3) at (2.5,-4.2) {Voronoi cell in $\dim = 3$};
      \node[left,inner sep=3mm,draw] (cvp3b) at (0,-4.8) {Closest vectors in $\dim = 3$};

      \node[left,inner sep=3mm,draw] (cvpca) at (0,-7.0) {Closest vectors in $\dim = d-1$};
      \node[right,inner sep=3mm,draw] (vcc) at (2.5,-7.6) {Voronoi cell in $\dim = d-1$};
      \node[left,inner sep=3mm,draw] (cvpcb) at (0,-8.2) {Closest vectors in $\dim = d-1$};
      \node[left,inner sep=3mm,draw] (cvpda) at (0,-9.4) {Closest vectors in $\dim = d$};
      \node[right,inner sep=3mm,draw] (vcd) at (2.5,-10.0) {Voronoi cell in $\dim = d$};

      \draw[->] (cvp1) -- (cvp2a);
      \draw[->] (cvp2a.east) -- (vc2);
      \draw[->] (vc2) -- (cvp2b.east);
      \draw[->] (cvp2b) -- (cvp3a);
      \draw[->] (cvp3a.east) -- (vc3);
      \draw[->] (vc3) -- (cvp3b.east);

      \draw[->] (cvpca.east) -- (vcc);
      \draw[->] (vcc) -- (cvpcb.east);
      \draw[->] (cvpcb) -- (cvpda);
      \draw[->] (cvpda.east) -- (vcd);
    \end{tikzpicture}
  \end{center}
  \caption{Iteratively building Voronoi cells of larger dimension.}
  \label{fig:voronoi-cell-dimension-reduction}
\end{figure}

Let us recall the nearest plane method, see Figure~\ref{fig:nearest-plane-approximation}.
In Section~\ref{fig:nearest-plane-approximation},
we saw that if the basis $B$ is LLL-reduced,
there is a vector $x \in \Lambda$ such that
\[
  \|x-t\|_2^2 \leq 2^{d-2} \|b_d^\star\|_2^2
\]
Since $\|b_d^\star\|_2$ is the distance between adjacent lattice hyperplanes,
it follows that a closest vector to $t$ must lie,
if not on the nearest lattice hyperplane,
then on one of the $2 \cdot 2^{(d-2)/2} = 2^{d/2}$ lattice hyperplanes closest to $t$.
By projecting $t$ orthogonally onto each of those hyperplanes in turn
and solving a closest vector problem in $\Lambda'$,
we are guaranteed to find a closest vector.

\begin{theorem}
  The Voronoi relevant vectors of a lattice can be computed in time $2^{O(d)} \poly(b)$.
  As a consequence, the shortest and closest vector problems can be solved in the same asymptotic
  running time.
\end{theorem}






\section*{Exercises}

\begin{enumerate}
  \item
    \begin{enumerate}[(a)]
    \item Show that $(\frac{1}{2} \Lambda) / \Lambda \cong (\Z_2)^d$.

    \item Show that $|(\Z_2)^d| = 2^d = \frac{\det\Lambda}{\det \frac{1}{2} \Lambda}$.
    \end{enumerate}

  \item
    Define a suitable parity function on $\Lambda$ such that:
    \begin{enumerate}[(i)]
      \item A lattice vector of parity $0$ cannot define a facet of $\cV_\Lambda$.

      \item Given two vectors $x, y \in \Lambda$, $y \neq \pm x$, of the \emph{same} parity,
        show that at least one of them does not define a facet of $\cV_\Lambda$.
    \end{enumerate}
    Conclude with an alternative proof of the statement of Corollary~\ref{cor:number-of-voronoi-relevant}:
    there can be at most $2 \cdot (2^d - 1)$ Voronoi relevant vectors.
\end{enumerate}

% Copyright 2013 Nicolai Hähnle <nhaehnle@gmail.com>
%
% This work is licensed under the Creative Commons Attribution-ShareAlike 3.0
% Unported License, see http://creativecommons.org/licenses/by-sa/3.0/
%
% Among other things, this means that yes, you may take e.g. illustrations from
% the book and use them in your own work. However, (a) you must give proper
% attribution by naming me as its original author and (b) you must make your
% derivative work available under the same or similar license terms.
%
% See the Creative Commons website for the exact licensing terms.

\chapter{Dual lattices and Fourier analysis}
\label{chapter:dual-lattices}

Consider the problem of integer programming or, more generally, lattice programming:
given a closed convex set $K$ and a lattice $\Lambda$,
decide whether there exists a lattice point $x \in K \cap \Lambda$.
A natural approach to deciding this problem
is to slice $K$ along translates of a lattice hyperplane,
analogous to the nearest-plane approach to the closest vector problem.
Each of the slices intersecting $K$ leads to an integer programming problem
of lower dimension.
\begin{center}
  \begin{tikzpicture}
    \draw[thick,fill=black!10]
      (0,0) -- (2,-1.3) -- (5,-0.2) node[below right] {$K$} arc[start angle=-50, end angle=10, radius=2cm]
      -- (3,2) -- (1,1.8) -- cycle;

    \draw (6,1.5) node[right] {$y^Tx = \alpha \in \Z$};
    \draw[->] (-0.5,0.1) -- node[left,near end] {$y$} +(0.1,-0.5);
    \clip (-1,-2) rectangle (6,2.5);
    \foreach \t in {-3,-2,-1,0,1,2}
      \draw ($(-1,0) + \t*(0,1.2)$) -- +(7,1.4);
  \end{tikzpicture}
\end{center}
These translates of lattice hyperplanes are defined by equations $y^Tx = \alpha \in \Z$,
where $y \in \R^d$ satisfies $y^Tx \in \Z$ for all $x \in \Lambda$.
This is one justification for the definition of \emph{dual lattices},
which we study in this chapter.

The running time of this particular approach to integer programming
depends strongly on how many lattice hyperplanes intersect $K$.
Fourier analysis neatly relates a lattice and its dual,
which allows us to bound the number of lattice hyperplanes that must be investigated.


\section{The dual lattice}

\begin{definition}
  Let $\Lambda \subset \R^d$ be a full dimensional lattice.
  Its \emph{dual lattice} is given by
  \[
    \Lambda^\star := \{ y \in \R^d ~:~ \forall x \in \Lambda:\, y^Tx \in \Z \}
  \]
\end{definition}

\begin{lemma}
  \label{lemma:dual-basis}
  Let $B \in \R^{d \times d}$ be a basis of $\Lambda$.
  Then $\Lambda^\star = \Lambda(B^{-T})$.
  In particular, $\Lambda^\star$ is a lattice.
\end{lemma}
\begin{proof}
  Let $c_1, \ldots, c_d$ be the columns of $B^{-T}$.
  \[
    \begin{array}{|c|}
      \hline \quad c_1^T \quad  \\\hline
      \vdots \\\hline
      c_d^T \\\hline
    \end{array}
    \cdot
    \begin{array}{|c|c|c|}
      \hline  & &\\[0.5em]
      b_1 & \dots & b_d \\
      & & \\[0.5em]\hline
    \end{array}
    =
    \begin{array}{|lcr|}
      \hline 1 & & 0 \\
       & \ddots & \\
      0 & & 1 \\\hline
    \end{array}
  \]
  Let $y \in \Lambda^\star$.
  Since the $c_1, \ldots, c_d$ form a basis of $\R^d$,
  we can write
  \[
    y = \alpha_1 c_1 + \dots + \alpha_d c_d,\, \alpha_j \in \R
  \]
  It suffices to show that all $c_j \in \Z$,
  which follows from
  \[
    \Z \ni y^T b_j = \alpha_1 c_1^T b_j + \dots + \alpha_d c_d^T b_j = \alpha_j.
  \]
  Now suppose $y \in \Lambda(B^{-T})$, that is,
  we can write
  \[
    y = \alpha_1 c_1 + \dots + \alpha_d c_d,\, \alpha_j \in \Z
  \]
  Let $x \in \Lambda$, that is,
  \[
    x = \beta_1 b_1 + \dots + \beta_d b_d,\, \beta_j \in \Z
  \]
  Then
  \[
    y^Tx = \alpha_1 \beta_1 + \dots + \alpha_d \beta_d \in \Z
  \]
  That is, $y^Tx \in \Z$ for all $x \in \Lambda$,
  hence $y \in \Lambda^\star$ by definition.
\end{proof}

\begin{corollary}
  \label{corollary:transformed-dual-lattice}
  Let $\Lambda \subset \R^d$ be a full-dimensional lattice.
  \begin{enumerate}
    \item $\det \Lambda^\star = \frac{1}{\det \Lambda}$.

    \item $(\Lambda^\star)^\star = \Lambda$.

    \item Let $M \in \R^{d\times d}$ be an invertible matrix.
      Then $(M \Lambda)^\star = M^{-T} \Lambda^\star$.

    \item $(\alpha \Lambda)^\star = \frac{1}{\alpha} \Lambda^\star$ for $\alpha > 0$.
  \end{enumerate}
\end{corollary}

Intuitively, a dense lattice has a sparse dual and vice versa.
Two aspects of this connection are formalized in the Corollary,
but it can also be seen in the lattice hyperplanes corresponding to dual lattice vectors.
For a given $y \in \Lambda^\star$,
the distance between adjacent lattice hyperplanes $y^Tx = \alpha$ and $y^Tx = \alpha + 1$
for $\alpha \in \Z$ is $1 / \|y\|_2$.
In a dense lattice, lattice hyperplane translates must lie close to each other,
which means that dual lattice vectors must be long.
We will develop more statements of this form throughout this chapter.



\section{Successive minima and covering radius}

Let us define some measures of the overall sparsity of a lattice.

\begin{definition}
  Let $\Lambda \subset \R^d$ be a full-dimensional lattice.
  The \emph{successive minima} $\lambda_1, \ldots, \lambda_d$ of $\Lambda$ are
  \[
    \lambda_j(\Lambda) := \min\{ r > 0 ~:~ \dim( \Lambda \cap B(0,r) ) \geq j \}
  \]
  where the dimension is the dimension of the linear span of the lattice points
  of norm at most $r$.

  The \emph{covering radius} of $\Lambda$ is the maximal distance from the lattice:
  \[
    \mu(\Lambda) := \max_{p \in \R^d} d(p, \Lambda)
  \]
  When the lattice is clear from the context,
  we write $\lambda_1, \ldots, \lambda_d$ and $\mu$.
  Furthermore, we write $\lambda_j^\star$ and $\mu^\star$ for the corresponding
  quantities of the dual lattice.
\end{definition}

The covering radius can be equivalently defined as the smallest radius $r$
such that the union of balls of radius $r$ around each lattice point
covers the entire space -- hence its name.

\begin{example}
  \label{example:rect-lattice-minima}
  The lattice $\Lambda := \Lambda \begin{pmatrix} 1 & 0 \\ 0 & 3 \end{pmatrix}$
  satisfies $\lambda_1 = 1$, $\lambda_2 = 3$, and $\mu = \sqrt{10}/2$.
  \begin{center}
    \begin{tikzpicture}
      \foreach \x in {-2,-1,...,2.1}
        \foreach \y in {-1,0,1}
          \fill ($\x*(0.5,0) + \y*(0,1.5)$) circle[radius=2pt];

      \draw (0,0) node[below] {$0$};

      \draw (0.25,0.75) circle[radius=0.76cm];
      \fill (0.25,0.75) circle[radius=2pt] node[right] {$p$};
    \end{tikzpicture}
  \end{center}
\end{example}

As the example shows,
$\lambda_1$ is the length of a shortest vector,
but $\lambda_2$ need not be the length of a ``second-shortest'' vector.
Instead, it is the length of a shortest vector among all vectors that are
linearly independent from a shortest vector.
In general,
we can choose linearly independent vectors $v_1, \ldots, v_d \in \Lambda$ with $\|v_j\|_2 = \lambda_j$.
Such a set of vectors is called a set of \emph{shortest independent vectors} of the lattice.
An exercise shows that such a set need \emph{not} be a basis of the lattice.

\begin{lemma}
  \label{lemma:mu-lambda-d}
  Let $\Lambda$ be a $d$-dimensional lattice.
  Then $\mu \geq \lambda_d / 2$.
\end{lemma}
\begin{proof}
  All lattice points in the interior of the ball of radius $\lambda_d$ around the origin
  lie in a hyperplane $H$.
  A point at distance $\lambda_d/2$ from $H$ has distance at least $\lambda_d/2$ to the lattice,
  see Figure~\ref{fig:mu-vs-lambda-d}.
\end{proof}
\begin{figure}
  \begin{center}
    \begin{tikzpicture}
      \fill (0,0) circle[radius=2pt];
      \draw (0,0) circle[radius=2cm];
      \draw[thick] (20:-3) -- (20:3) node[below right] {$H$};

      \draw (0,0) -- node[right] {$\lambda_d$} (110:2);
      \draw (110:1) circle[radius=1cm];
    \end{tikzpicture}
  \end{center}
  \caption{The proof that $\mu \geq \lambda_d / 2$.}
  \label{fig:mu-vs-lambda-d}
\end{figure}

The lattice $\Z$ shows that the inequality of this Lemma is tight.
Let us now relate quantities of a lattice and its dual.
The following Lemma shows that a lattice and its dual cannot be too dense at the same time.

\begin{lemma}
  \label{lemma:transference-lower-bound}
  Let $\Lambda \subset \R^d$ be a full-dimensional lattice.
  Then $\lambda_1^\star \cdot \lambda_d \geq 1$,
  and hence $\lambda_1^\star \cdot \mu \geq 1/2$.
\end{lemma}
\begin{proof}
  Let $y \in \Lambda^\star$ be a shortest vector
  and let $v_1, \ldots, v_d \in \Lambda$ be shortest independent vectors.
  Since the $v_j$ form a basis of $\R^d$,
  we must have $y^T v_k \neq 0$ for at least one $k$.
  Using the Cauchy-Schwarz inequality and the fact that $y^T v_k \in \Z$,
  we can compute
  \[
    1 \leq |y^T v_k| \leq \|y\|_2 \|v_k\|_2 = \lambda_1^\star \lambda_k \leq \lambda_1^\star \lambda_d \qedhere
  \]
\end{proof}

\begin{example}
  Consider $\Lambda = \Lambda \begin{pmatrix} 1 & 0 \\ 0 & M \end{pmatrix}$ for $M > 1$.
  We have $\Lambda^\star = \Lambda \begin{pmatrix} 1 & 0 \\ 0 & M^{-1} \end{pmatrix}$
  and therefore
  \begin{align*}
    \lambda_1 &= 1 \\
    \lambda_d &= M \\
    \lambda_1^\star &= M^{-1} \\
    \lambda_d^\star &= 1
  \end{align*}
  That is, the inequality of Lemma~\ref{lemma:transference-lower-bound} is tight,
  both when applied to $\Lambda$ and when applied to $\Lambda^\star$.
\end{example}

We are interested in analogous statements showing
that a lattice and its dual cannot be too \emph{sparse} at the same time.
That is, we would like to have an \emph{upper bound} on a product of the form $\lambda_1^\star \cdot \lambda_d$.
Some bounds of this type are shown in the exercises of this chapter,
using Minkowski's theorem and LLL-reduced bases.
We will derive a much stronger bound in Section~\ref{sec:transference-bound-banaszczyk}.



\section{Lattice width and Flatness theorems}

We will now tie dual lattices and the quantities defined in the last section
together with the approach to integer programming based on lattice hyperplanes.

\begin{definition}
  Let $K \subset \R^d$ be a convex body.
  Let $\Lambda$ be a full-dimensional lattice and let $y \in \Lambda^\star$.
  The \emph{lattice width of $K$ with respect to $y$} is
  \[
    w_y(K, \Lambda) := \max_{x \in K} y^T x - \min_{x \in K} y^T x
  \]
  The \emph{lattice width of $K$} is
  \[
    w(K, \Lambda) := \min_{y \in \Lambda^\star \setminus 0} w_y(K,\Lambda)
  \]
  We write $w_y(K)$ or $w(K)$ when the lattice is clear from the context.
\end{definition}

The lattice width with respect to $y \in \Lambda^\star$
describes the distance in terms of lattice hyperplanes between parallel planes
sandwiching the convex body.
This width is particularly easy to compute for a ball $B(z,r)$,
see Figure~\ref{fig:lattice-width-ball}:
\begin{align*}
  w_y(B(z,r)) &= \max_{x \in B(z,r)} y^Tx - \min_{x \in B(z,r)} y^Tx \\
   &= y^T (z + r \frac{y}{\|y\|_2}) - y^T (z - r \frac{y}{\|y\|_2}) \\
   &= 2r \frac{y^T y}{\|y\|_2} = 2r \|y\|_2
\end{align*}
\begin{figure}
  \begin{center}
    \begin{tikzpicture}
      \draw[thick,fill=black!10] (0,0) circle[radius=1.5cm];
      \fill (0,0) circle[radius=2pt] node[left] {$z$};
      \draw (-30:1.5) +(60:-2) -- +(60:2);
      \draw (-30:-1.5) +(60:-2) -- +(60:2);
      \draw[->] (-30:1.5) ++(60:1) -- node[above,near end] {$y$} +(-30:0.6);
      \draw (0,0) -- node[above] {$r$} (-30:1.5);
    \end{tikzpicture}
  \end{center}
  \caption{The lattice width of a ball with respect to $y \in \Lambda^\star$.}
  \label{fig:lattice-width-ball}
\end{figure}

\begin{lemma}
  The lattice width of a ball of radius $r$ is $2r\lambda_1^\star$.
\end{lemma}
\begin{proof}
  \[
    w(B(z,r)) = \min_{y \in \Lambda^\star \setminus 0} w_y(B(z,r))
      = \min_{y \in \Lambda^\star \setminus 0} 2r \|y\|_2 = 2r \lambda_1^\star
    \qedhere
  \]
\end{proof}

Since the lattice width is invariant under linear transformations (see the exercises),
this result easily extends to ellipsoids.
Intuitively, we can find the lattice width of an ellipsoid $E$
by finding a matrix $A$ such that $A \cdot E$ is a ball
and then computing a shortest vector in $A^{-T} \cdot \Lambda^\star$,
the dual lattice of $A \cdot \Lambda$.

\begin{theorem}[Flatness Theorem for Ellipsoids]
  \label{thm:flatness-ellipsoids}
  Let $d \in \N$.
  Suppose there is a constant $c_d$ such that for every full-dimensional lattice $\Lambda \subset \R^d$
  one has $\lambda_1^\star \cdot \mu \leq c_d$.
  Then every ellipsoid $E$ with $w(E) \geq 2 c_d$ contains a lattice point.
\end{theorem}
\begin{proof}
  By the above remark, we only need to consider the case where $E$ is a ball.
  We have
  \[
    2 \lambda_1^\star \cdot \mu \leq 2 c_d \leq w(E) = 2r \lambda_1^\star,
  \]
  where $r$ is the radius of $E$.
  This implies $\mu \leq r$.
  Since $\mu$ is the maximal distance of any point from the lattice,
  it follows that there is a lattice point with distance less than $r$ from the center of $E$.
\end{proof}

This suggests an analysis of the approach to integer programming
that we sketched in the beginning of this chapter,
at least in the case where $K$ is an ellipsoid:
Determine $w(K)$ using a shortest vector problem in an appropriately transformed dual lattice.
If $w(K) > 2c_d$, we can assert that $K$ contains a lattice point.
Otherwise, there are at most $\lfloor 2c_d + 1 \rfloor$
lattice hyperplanes in which we recursively solve a lower dimensional problem.
This leads to a running time of essentially $2^d c_d c_{d-1} \cdots c_1$.
In the remainder of this chapter,
we will show that we can choose $c_d = d$,
which leads to a running time for integer programming of essentially $2^{O(d \log d)}$.

For now,
let us remark that integer programming on ellipsoids is not very interesting in itself
because it is equivalent to a closest vector problem
with respect to a distorted $\ell_2$ norm.
We already saw how to solve this problem more efficiently in Chapter~\ref{chapter:voronoi-cell}.
Unlike the approach via Voronoi diagrams, however,
this new approach can be generalized to other convex bodies.

The key to the generalization is a classical result from convex geometry
about approximation of convex bodies by ellipsoids, see Figure~\ref{fig:loewner-john}.
For a proof of this result, see e.g. the exposition in~\cite{MR1491097}.
Note that the scaling factor of $d$ in the following Theorem is best possible,
since it is tight e.g. for simplices.

\begin{figure}
  \begin{center}
    \begin{tikzpicture}
      \draw (0,0) circle[radius=1.5cm];
      \draw (0,0) circle[radius=3cm];

      \path[name path=face1] (30:1.5) +(120:-3) -- +(120:3);
      \path[name path=faceA] (100:2.6) +(25:-3) -- +(25:3);
      \path[name path=face2] (150:1.5) +(60:-3) -- +(60:3);
      \path[name path=faceB] (180:1.9) + (90:-3) -- +(90:3);
      \path[name path=face3] (270:1.5) +(-3,0) -- +(3,0);
      \path[name path=faceC] (-45:2.1) +(45:-3) -- +(45:3);

      \path[name intersections={of=face1 and faceA}] (intersection-1) coordinate (a);
      \path[name intersections={of=face2 and faceA}] (intersection-1) coordinate (b);
      \path[name intersections={of=face2 and faceB}] (intersection-1) coordinate (c);
      \path[name intersections={of=face3 and faceB}] (intersection-1) coordinate (d);
      \path[name intersections={of=face3 and faceC}] (intersection-1) coordinate (e);
      \path[name intersections={of=face1 and faceC}] (intersection-1) coordinate (f);

      \draw[thick] (a) -- (b) -- (c) -- (d) -- (e) -- (f) -- cycle;
    \end{tikzpicture}
  \end{center}
  \caption{Approximating a convex body by ellipsoids as per Theorem~\ref{thm:loewner-john}.}
  \label{fig:loewner-john}
\end{figure}

\begin{theorem}[Löwner-John ellipsoid]
  \label{thm:loewner-john}
  Let $K \subset \R^d$ be a convex body of positive volume
  and let $E \subseteq K$ be an ellipsoid of maximal volume.
  Then
  \begin{enumerate}
    \item $E$ is unique and
    \item $K \subset d \star E$, where $d \star E$ is the scaling of $E$ by a factor of $d$ around its center.
  \end{enumerate}
\end{theorem}

\begin{corollary}[Flatness Theorem]
  Let $d \in \N$.
  Suppose there is a constant $c_d$ such that
  for every full-dimensional lattice $\Lambda \subset \R^d$
  one has $\lambda_1^\star \cdot \mu \leq c_d$.
  Then every convex body $K$ with $w(K) \geq 2d \cdot c_d$ contains a lattice point.
\end{corollary}
\begin{proof}
  Let $E \subseteq K \subseteq d \star E$ as in Theorem~\ref{thm:loewner-john}.
  Since
  \[
    d \cdot w(E) = w(d \star E) \geq w(K) \geq 2d \cdot c_d
  \]
  we have $w(E) \geq 2c_d$,
  so there is a lattice point $x \in \Lambda \cap E \subseteq \Lambda \cap E$
  by Theorem~\ref{thm:flatness-ellipsoids}.
\end{proof}

The best possible \emph{flatness constant} that can be derived with this approach is $\Theta(d^2)$.
A better flatness constant has been shown using more advanced techniques~\cite{MR1854250, MR1755679}:
if $K$ is an arbitrary convex body with $w(K) > \Omega(d^{4/3} \log^c d)$,
then $K$ contains a lattice point.
For centrally symmetric empty convex bodies
and for polytopes with polynomially many vertices or facets,
the flatness constant is known to be $O(d \log d)$~\cite{MR1410163,MR1854250}.



\section{Fourier series}

Our goal for the remainder of the chapter is to show a \emph{transference bound}
$\lambda_1^\star \cdot \mu \leq d$.
The covering radius $\mu$ is defined as the maximum of the $\Lambda$-periodic function $d(p,\Lambda)$,
so we will use Fourier analysis to study such functions.

\begin{definition}
  A function $f: \R^d \to \C$ is $\Lambda$-periodic if $f(p + x) = f(p)$ for all $p \in \R^d$
  and $x \in \Lambda$.
\end{definition}

One can say that
Fourier analysis is the study of a change of coordinates between two orthonormal bases
of a space of functions.
One of the bases is ``the pointwise basis'' while the other basis is in terms of simple wave functions.
Such language is to be taken with a grain of salt when talking about spaces of functions,
but it should serve as a reasonable guide to intuition.

Talking about orthonormality requires a scalar product on $\Lambda$-periodic functions:
\[
  \langle f, g \rangle_\Lambda := \frac{1}{\det \Lambda} \int_{\cP_B} f(p) \overline{g(p)} \,dp
\]
where $B$ is some basis of $\Lambda$ and $\overline{z}$ is the complex conjugate of $z \in \C$.
As an orthonormal basis of wave functions,
we will use
\[
  \omega_y(p) := e^{2\pi i p^T y}
\]
for $y \in \Lambda^\star$.
Let us establish some basic facts about $\langle \cdot, \cdot \rangle_\Lambda$ and the $\omega_y$.

\begin{remark}
  In general,
  we will not be precise about the exact integrability (and often continuity) conditions on $f$ and $g$.
\end{remark}

\begin{lemma}
  $\langle \cdot, \cdot \rangle_\Lambda$ is a scalar product on
  sufficiently nice $\Lambda$-periodic functions.
  Furthermore, for every half-open polytope $P \subset \R^d$ such that
  the translates $x + P$, $x \in \Lambda$, form a tiling of $\R^d$
  (that is, $\R^d = \bigcup_{x\in \Lambda} x + P$ is a disjoint union),
  one has
  \[
    \langle f, g \rangle_\Lambda = \frac{1}{\det \Lambda} \int_{P} f(p) \overline{g(p)} \,dp.
  \]
  In particular, the definition of $\langle \cdot, \cdot \rangle_\Lambda$
  is independent of the choice of basis $B$.
\end{lemma}
\begin{proof}
  One easily checks that $\langle \cdot, \cdot \rangle_\Lambda$
  is a scalar product, that is, it satisfies linearity, conjugate symmetry, and positive definiteness.

  For the second part,
  the intuitive idea is that we can slice $P$ along the tiling defined by $\cP_B$
  and then reassemble the pieces of $P$ into a disjoint union
  equal to $\cP_B$.
  This is similar to the proof of Theorem~\ref{thm:blichfeldt}.

  We define $P_x := P \cap (x + \cP_B)$ for $x \in \Lambda$
  and note that $P$ is the disjoint union of the $P_x$.
  It is straightforward to show that $\cP_B$ is the disjoint union of the $P_x - x$.
  We can use this to compute
  \begin{align*}
    \int_P f(p) \overline{g(p)} \,dp &= \sum_{x \in \Lambda} \int_{P_x} f(p) \overline{g(p)} \,dp
      = \sum_{x \in \Lambda} \int_{P_x - x} f(p) \overline{g(p)} \,dp
      = \int_{\cP_B} f(p) \overline{g(p)} \,dp
  \end{align*}
  For the second equation, we used the $\Lambda$-periodicity of $f$ and $g$.
  Note that even though we operate with infinite sums,
  there are no issues of convergence because only finitely many summands are non-zero.
\end{proof}

\begin{lemma}
  \label{lemma:omega-orthonormal}
  The $\omega_y$ form an orthonormal system of $\Lambda$-periodic functions.
\end{lemma}
\begin{proof}
  First, we check that
  \[
    \omega_y(p + x) = e^{2\pi i (p + x)^T y}
      = e^{2\pi i p^Ty} \cdot \underbrace{e^{2\pi i \overbrace{x^Ty}^{\in \Z}}}_{=1} = \omega_y(p)
  \]
  for $y \in \Lambda^\star$ and $x \in \Lambda$.
  Furthermore,
  \begin{align*}
    \langle \omega_y, \omega_y \rangle &= \frac{1}{\det \Lambda} \int_{\cP_B} \omega_y(p) \overline{\omega_y(p)} \,dp \\
     &= \frac{1}{\det\Lambda} \int_{\cP_B} 1 \,dp = 1
  \end{align*}
  Finally, let $y \neq z \in \Lambda^\star$. Then
  \begin{align*}
    \langle \omega_y, \omega_z \rangle
      &= \frac{1}{\det \Lambda} \int_{\cP_B} e^{2\pi i p^T(y - z)} \,dp
  \end{align*}
  Let $\Lambda' := \{ x \in \Lambda ~:~ x^T (y - z) = 0 \}$
  and let $P'$ be a fundamental parallelepiped of $\Lambda'$.
  Let $k \in \N_{\geq 1}$ be minimal such that the hyperplane $x^T (y - z) = k$
  contains a lattice point.
  Then $P := P' + [0,w)$, $w := k \cdot \frac{y - z}{\|y-z\|_2^2}$,
  is a fundamental region of $\Lambda$, see Figure~\ref{fig:omega-orthogonal},
  and we can rewrite the integral as.
  \begin{align*}
    \langle \omega_y, \omega_z \rangle
      &= \frac{1}{\det \Lambda} \int_{P} e^{2\pi i p^T(y - z)} \,dp \\
      &= \frac{1}{\det \Lambda} \int_{P'} e^{2\pi i p^T(y - z)}
        \underbrace{\int_{[0,w)} e^{2\pi i q^T(y-z)} \,dq}_{=0} \,dp = 0
  \end{align*}
  The last equation follows from the fact that $w^T (y - z) = k$ is a positive integer,
  and the inner integral can therefore be written as an integral over a closed path in the complex plane.
\end{proof}
\begin{figure}
  \begin{center}
    \begin{tikzpicture}
      \coordinate (w) at (1.8,0.6);
      \coordinate (w') at ($(w) + (-0.4,1.2)$);
      \fill[black!10] (0,0) -- (w) -- (w') -- (-0.4,1.2) -- cycle;
      \draw[very thick] (0,0) -- node[left] {$P'$} (-0.4,1.2);
      \draw (0,0) -- (w);
      \draw (-0.4,1.2) -- (w');

      \fill (0,0) circle[radius=2pt] node[left] {$0$};

      \foreach \x in {0,1,...,3} {
        \draw ($\x*(2,0)$) +(0.5,-1.5) -- +(-1.0,3);
        \foreach \y in {-1,0,...,2.01}
          \fill ($\x*(2,0) + \y*(-0.4,1.2)$) circle[radius=2pt];
      }

      \draw[fill=white] (w) circle[radius=2pt] node[right] {$w$};
      \draw (0.7,0.9) node {$P$};

      \draw (0.5,-1.5) node[below] {$\scriptstyle x^T(y-z) = 0$};
      \draw (2.5,-1.5) node[below] {$\scriptstyle x^T(y-z) = 2$};

      \draw[->] (2.2,-0.6) -- node[below,very near end] {$y-z$} +(1.2,0.4);

      \foreach \x in {-0.5,0.5,1.5,...,3.501}
        \draw[help lines] ($\x*(2,0)$) +(0.5,-1.5) -- +(-1.0,3);

    \end{tikzpicture}
  \end{center}
  \caption{Integrating parallel to $y-z$ in the proof of Lemma~\ref{lemma:omega-orthonormal}.}
  \label{fig:omega-orthogonal}
\end{figure}

\begin{definition}
  Let $f: \R^d \to \C$ be $\Lambda$-periodic.
  The \emph{Fourier series} of $f$ is $\hat f: \Lambda^\star \to \C$ where
  \[
    \hat f(y) := \langle f, \omega_y \rangle_\Lambda = \frac{1}{\det \Lambda} \int_{\cP_B} f(p) e^{-2\pi i p^T y} \,dp,
    \quad y \in \Lambda^\star
  \]
\end{definition}
\begin{theorem}
  For sufficiently nice $\Lambda$-periodic $f$, one has $f = \sum_{y \in \Lambda^\star} \hat f(y) \omega_y$.
\end{theorem}




\section{The distance from a lattice and Fourier transform}

The function $p \mapsto d(p,\Lambda)$,
which we might want to use to study the covering radius $\mu$,
is $\Lambda$-periodic.
However,
it is unwieldy for Fourier analysis since it is not even differentiable.
Let us instead consider the function $g_\Lambda: \R^d \to \R$ defined by
\[
  g_\Lambda(p) := \sum_{x \in \Lambda} e^{-\pi \|p - x\|_2^2}
\]
This function is a sum of normal distributions centered at each lattice point:
\begin{center}
  \begin{tikzpicture}
    \clip (-2,-1.5) rectangle (2,1.5);
    \foreach \x in {-3,-2,...,3.01}
      \foreach \y in {-2,-1,...,2.01} {
        \fill ($\x*(1.4,0.1) + \y*(0.2,1.2)$) circle[radius=2pt];
        \draw[help lines] ($\x*(1.4,0.1) + \y*(0.2,1.2)$)
          circle[radius=0.1cm]
          circle[radius=0.2cm]
          circle[radius=0.28cm]
          circle[radius=0.34cm]
          circle[radius=0.40cm]
          circle[radius=0.55cm];
      }
  \end{tikzpicture}
\end{center}
Intuitively, $g_\Lambda(p)$ should be quite small if $d(p,\Lambda) > \sqrt{d}$,
because then $e^{-\pi \|p - x\|_2^2} < e^{-\pi d}$ for every $x \in \Lambda$,
so it should be a useful tool for lattices with $\mu > \sqrt{d}$.

This is not quite true in absolute terms, however.
Consider the lattice $\Lambda := \Lambda \begin{pmatrix} \varepsilon & 0 \\ 0 & C \end{pmatrix}$
for $\varepsilon > 0$ small and $C > 0$ large, see Figure~\ref{fig:g-sparse-dense-lattice}.
There is a point $p$ at distance $C/2$ of the lattice,
but $g_\Lambda(p) \to \infty$ as $\varepsilon \to 0$.
That is, we will have to measure $g_\Lambda(p)$ relative to some other quantity.

\begin{figure}
  \begin{center}
    \begin{tikzpicture}
      \foreach \x in {-10,-9,...,10.01}
        \foreach \y in {0,1}
          \fill ($\x*(0.24,0) + \y*(0,2)$) circle[radius=2pt];

      \fill (0.12,1) circle[radius=2pt] node[right] {$p$};
    \end{tikzpicture}
  \end{center}
  \caption{$g_\Lambda(p)$ can be arbitrarily large even for points that are far from the lattice.}
  \label{fig:g-sparse-dense-lattice}
\end{figure}

First, let us compute the Fourier series of $g$.
\begin{align*}
  \widehat{g_\Lambda}(y) &= \langle g_\Lambda, \omega_y \rangle_\Lambda
    = \frac{1}{\det\Lambda} \int_{\cP_B} g_\Lambda(p) e^{-2\pi i p^T y} \,dp \\
    &= \frac{1}{\det\Lambda} \int_{\cP_B} \sum_{x \in \Lambda} e^{-\pi \|p - x\|_2^2} e^{-2\pi i p^T y} \,dp \\
    &= \frac{1}{\det\Lambda} \int_{\cP_B} \sum_{x \in \Lambda} e^{-\pi \|p + x\|_2^2} e^{-2\pi i (p + x)^T y} \,dp \\
    &= \frac{1}{\det\Lambda} \sum_{x \in \Lambda} \int_{x + \cP_B} e^{-\pi \|p\|_2^2} e^{-2\pi i p^T y} \,dp \\
    &= \frac{1}{\det\Lambda} \int_{\R^d} e^{-\pi \|p\|_2^2} e^{-2\pi i p^T y} \,dp
\end{align*}
This motivates the introduction of the second main tool from Fourier analysis.

\begin{definition}
  Let $f:\R^d \to \C$ with $\int |f| < \infty$.
  The \emph{Fourier transform} of $f$ is $\tilde f: \R^d \to \C$, where
  \[
    \tilde f(y) := \int_{\R^d} f(p) e^{-2\pi i p^Ty} \,dp
  \]
\end{definition}
\begin{theorem}
  For sufficiently nice $f: \R^d \to \C$, one has $f(p) = \int_{\R^d} \tilde f(y) e^{2\pi i p^Ty} \,dy$.
\end{theorem}

\begin{fact}
  The function $\rho: \R^d \to \R$, $\rho(x) := e^{-\pi \|x\|_2^2}$
  is a fix point of the Fourier transform.
  That is, $\tilde \rho = \rho$.
\end{fact}

\begin{lemma}[Fourier coefficients of $g_\Lambda$]
  \label{lemma:fourier-coeff-g}
  $\widehat{g_\Lambda}(y) = \det\Lambda^\star \rho(y)$.
\end{lemma}
\begin{proof}
  We continue the computation from above:
  \begin{align*}
    \widehat{g_\Lambda}(y) &= \frac{1}{\det\Lambda} \int_{\R^d} \rho(p) e^{-2\pi i p^T y} \,dp \\
      &= \det\Lambda^\star \tilde\rho(y) = \det\Lambda^\star \rho(y) \qedhere
  \end{align*}
\end{proof}

As an aside,
Lemma~\ref{lemma:fourier-coeff-g} is related to the more general \emph{Poisson summation formula}.
For \emph{any} function $f(p) := \sum_{x \in \Lambda} h(p+x)$, where $h : \R^d \to \C$ is sufficiently nice,
the same kind of computation leads to
\[
  \hat f(y) = \det\Lambda^\star \tilde h(y)
\]
In particular, we get
\[
  h(\Lambda) = \sum_{x\in\Lambda} h(x)
    = f(0) = \sum_{y \in \Lambda^\star} \hat f(y)
    = \det\Lambda^\star \sum_{y \in \Lambda^\star} \tilde h(y) = \det\Lambda^\star \tilde h(\Lambda^\star).
\]
In any case, knowing the Fourier coefficients of $g_\Lambda$
opens the door to some new observations.
For example,
\[
  g_\Lambda(p) = \sum_{y \in \Lambda^\star} \underbrace{\widehat{g_\Lambda}(y)}_{> 0} \underbrace{e^{2\pi i p^T y}}_{|\cdot| = 1}
  \leq \sum_{y \in \Lambda^\star} \widehat{g_\Lambda}(y) = g_\Lambda(0)
\]
holds for all $p \in \R^d$.
Let us define
\[
  f_\Lambda(p) := g_\Lambda(p) / g_\Lambda(0),
\]
measuring $g_\Lambda$ relative to its value at a lattice point.
We can confirm our intuition that $f$ should be large for points that lie close to the lattice.

\begin{lemma}
  \label{lemma:f-large-close-to-lattice}
  For all $p \in \R^d$, one has $f_\Lambda(p) \geq e^{-\pi d(p,\Lambda)^2}$.
\end{lemma}
\begin{proof}
  For any $p \in \R^d$, we have
  \begin{align*}
    g_\Lambda(p) &= \sum_{x\in\Lambda} e^{-\pi \|p - x\|_2^2}
      = \frac{1}{2} \sum_{x\in\Lambda} \left( e^{-\pi \|p-x\|_2^2} + e^{-\pi \|p+x\|_2^2} \right) \\
      &= \sum_{x\in\Lambda} \frac{1}{2} \left( e^{-\pi (p^Tp - 2p^Tx + x^T x)} + e^{-\pi (p^Tp + 2p^Tx + x^Tx)} \right) \\
      &= e^{-\pi \|p\|_2^2} \sum_{x\in\Lambda} \frac{1}{2} \left( e^{-\pi (x - 2p)^T x} + e^{-\pi (x + 2p)^T x} \right) \\
      &\geq e^{-\pi \|p\|_2^2} \sum_{x\in\Lambda} e^{-\pi x^Tx} = e^{-\pi \|p\|_2^2} g_\Lambda(0),
  \end{align*}
  where the inequality follows from convexity.
  This implies the result because $\Lambda$-periodicity allows us
  to assume $\|p\|_2 = d(p,\Lambda)$ without loss of generality.
\end{proof}

In particular, if $d(p,\Lambda)$ is smaller than some constant,
then $f_\Lambda(p)$ is larger than a corresponding constant between $0$ and $1$.
On the other hand,
we will see that the inequality in Lemma~\ref{lemma:f-large-close-to-lattice}
is remarkably good even for points that are far away from the lattice.
In particular, $f_\Lambda(p)$ is exponentially small if $d(p,\Lambda) > \sqrt{d}$.

\begin{lemma}
  \label{lemma:f-small-close-to-lattice}
  For all $p \in \R^d$ and $k \in \N$, $k \geq 2$,
  one has $f(p) \leq k^d e^{-(1-1/k^2) \pi d(p,\Lambda)^2}$.

  When $d(p,\Lambda) \geq \sqrt{d}$, we have $f(p) \leq e^{-\pi d(p,\Lambda)^2 / 2}$.
\end{lemma}
\begin{proof}
  We compare the lattices $\Lambda$ and $\frac{1}{k} \Lambda$.
  \begin{align*}
    g_{\frac{1}{k}\Lambda}(p/k) &= \sum_{x\in\Lambda} e^{-\pi/k^2 \|p - x\|_2^2} \\
      &= \sum_{x \in\Lambda} e^{(1-1/k^2) \pi \|p - x\|_2^2} \cdot e^{-\pi \|p - x\|_2^2}
      \geq e^{(1-1/k^2) \pi d(p,\Lambda)^2} g_\Lambda(p)
  \end{align*}
  On the other hand,
  \begin{align*}
    g_{\frac{1}{k}\Lambda}(p/k)
      &\leq g_{\frac{1}{k}\Lambda}(0) \\
      &= \sum_{y \in (\frac{1}{k}\Lambda)^\star} \widehat{g_{\frac{1}{k}\Lambda}}(y)
      = \det (\frac{1}{k}\Lambda)^\star \sum_{y \in (\frac{1}{k}\Lambda)^\star} \rho(y)
      = k^d \det\Lambda^\star \sum_{y \in k\Lambda^\star} \rho(y) \\
      &\leq k^d \det\Lambda^\star \sum_{y \in \Lambda^\star} \rho(y)
      = k^d \sum_{y\in\Lambda^\star} \widehat{g_\Lambda}(y) = k^d g_\Lambda(0)
  \end{align*}
  and the first claim follows.

  The second claim follows by choosing $k = 2$,
  since for $d(p,\Lambda) \geq \sqrt{d}$:
  \begin{align*}
    2^d e^{-\frac{3}{4} \pi d(p,\Lambda)^2}
      &\leq e^{(\ln 2 - \frac{3}{4}\pi) d(p,\Lambda)^2} \leq e^{-\frac{1}{2} \pi d(p, \Lambda)^2}
  \end{align*}
  We used the fact that $\ln 2 < \pi/4$.
\end{proof}

\begin{remark}
  \begin{enumerate}
    \item 
      The condition that $k$ should be a natural number can be relaxed
      if we modify the proof using the Fourier transform of functions
      $\rho_k(p) = e^{-\pi \|p/k\|_2^2}$.

    \item
      Is it possible to efficiently compute a good approximation of $f(p)$?
      If yes, this might lead to a sub-exponential time algorithm that approximates
      the closest vector problem up to a factor of $\sqrt{d}$.
      The existence of such an algorithm is plausible
      because it is known that $\sqrt{d}$-approximate CVP lies in $NP \cap coNP$~\cite{MR2176561}.
      In fact, the presentation in this chapter is loosely inspired by that paper,
      which shows that an approximation of $f$ obtained by sampling Fourier coefficients
      may be used as a certificate that
      a given target point lies far from the lattice, and hence the problem lies in $coNP$.
  \end{enumerate}
\end{remark}








\section{Banaszczyk's transference bound}
\label{sec:transference-bound-banaszczyk}

Suppose that $\mu = \sqrt{d}$.
By Lemma~\ref{lemma:f-small-close-to-lattice},
there is a point $p_0$ for which $f(p_0) \leq e^{-\pi d / 2}$.
That is, $g_\Lambda(p)$ fluctuates quite dramatically.
On the other hand,
\begin{align*}
  g_\Lambda(p)
    &= \sum_{y \in \Lambda^\star} \widehat{g_\Lambda}(y) e^{2\pi i p^T y}
    = \det \Lambda^\star \sum_{y \in \Lambda^\star} \rho(y) e^{2\pi i p^T y},
\end{align*}
where the sum can be written as
\begin{align*}
  1 + \underbrace{\sum_{y \in \Lambda^\star \setminus 0} \rho(y) e^{2\pi i p^T y}}_{
    |\cdot| \leq \sum_{y \in \Lambda^\star} \rho(y)}
\end{align*}
It stands to reason that $\sum_{y \in \Lambda^\star} \rho(y) = g_{\Lambda^\star}(0) - 1$
is quite small when $\lambda_1^\star > \sqrt{d}$,
and hence $g_\Lambda(p)$ cannot fluctuate much.
This is indeed the case.
The Lemma formulates the required result for the primal rather than the dual lattice.

\begin{lemma}
  Let $\lambda_1 > \sqrt{d}$. Then $g_\Lambda(0) < 1 + 2 \cdot 2^{-d}$.
\end{lemma}
\begin{proof}
  The proof is analogous to Lemma~\ref{lemma:f-small-close-to-lattice}.
  \begin{align*}
   \sum_{y \in \Lambda \setminus 0} e^{-\pi/4 \|y\|_2^2}
      &= \sum_{y \in \Lambda \setminus 0} e^{-\pi \|y\|_2^2} \cdot e^{\frac{3}{4} \pi \overbrace{\|y\|_2^2}^{>d}}
      > e^{\frac{3}{4} \pi d} \sum_{y \in \Lambda \setminus 0} e^{-\pi \|y\|_2^2}
%      = e^{\frac{3}{4} \pi d} (g_\Lambda(0) - 1)
  \end{align*}
  On the other hand,
  \begin{align*}
    \sum_{y \in \Lambda \setminus 0} e^{-\pi/4 \|y\|_2^2}
      &= \sum_{y \in \frac{1}{2} \Lambda \setminus 0} e^{-\pi \|y\|_2^2}
 \\ & < g_{\frac{1}{2} \Lambda}(0)
      = \sum_{y \in (\frac{1}{2} \Lambda)^\star} \widehat{g_{\frac{1}{2} \Lambda}}(y)
 \\ & = \det(\frac{1}{2} \Lambda)^\star \sum_{y \in (\frac{1}{2} \Lambda)^\star} \rho(y)
      = 2^d \det \Lambda^\star \sum_{y \in 2\Lambda^\star} \rho(y)
 \\ & < 2^d \det \Lambda^\star \sum_{y \in \Lambda^\star} \rho(y)
      = 2^d \sum_{y \in \Lambda^\star} \widehat{g_\Lambda}(y)
      = 2^d g_\Lambda(0)
  \end{align*}
  Overall, we obtain
  \[
    2^d g_\Lambda(0) > e^{\frac{3}{4}\pi d} (g_\Lambda(0) - 1) > 4^d (g_\Lambda(0) - 1).
  \]
  We may rearrange as
  \[
    g_\Lambda(0) - 1 < 2^{-d} g_\Lambda(0)
  \]
  and finally get
  \[
    g_\Lambda(0) < \frac{1}{1 - 2^{-d}}
    = 1 + \frac{2^{-d}}{\underbrace{1 - 2^{-d}}_{\geq 1/2}} \leq 1 + 2 \cdot 2^{-d}.
    \qedhere
  \]
\end{proof}

Let us apply this result to the dual lattice and continue the line of thought from before.
If $\lambda_1^\star > \sqrt{d}$,
we now get
\[
  g_\Lambda(p) \geq \det \Lambda^\star (1 - (g_{\Lambda^\star}(0) - 1)) > \det \Lambda^\star (1 - 2 \cdot 2^{-d})
\]
for all $p \in \R^d$.
On the other hand,
\[
  g_\Lambda(0) = \det \Lambda^\star g_{\Lambda^\star}(0) < \det\Lambda^\star (1 + 2 \cdot 2^{-d}).
\]
To summarize,
\begin{align*}
  f(p) &= \frac{g_\Lambda(p)}{g_\Lambda(0)}
     > \frac{1 - 2 \cdot 2^{-d}}{1+2\cdot 2^{-d}}
     = 1 - \frac{4 \cdot 2^{-d}}{1 + 2 \cdot 2^{-d}}
     = 1 - \frac{2}{2^{d - 1} + 1} \geq \frac{1}{3}
\end{align*}
for $d \geq 2$.
On the other hand,
we exhibited a point with $f(p_0) \leq e^{-\pi d/2} < \frac{1}{3}$.
This is a contradiction.

\begin{theorem}
  $\lambda_1^\star \cdot \mu \leq d$.
\end{theorem}
\begin{proof}
  Scaling the lattice does not change $\lambda_1^\star \cdot \mu$,
  so we may assume $\mu = \sqrt{d}$.
  For $d = 2$, the discussion above shows a contradiction when $\lambda_1^\star > \sqrt{d}$.
  For $d = 1$, observe that the claim holds for $\Lambda = \Z$.
  The same scaling argument extends the claim to any other lattice.
\end{proof}










\section*{Exercises}

\begin{enumerate}
  \item
    Let $\Lambda \subset \R^d$ be a full-dimensional lattice.
    Show: a vector $y \in \Lambda^\star \setminus 0$ is primitive
    if and only if
    every lattice hyperplane $\{ y^T x = \alpha\}$, $\alpha \in \Z$, contains a point of $\Lambda$.

  \item Prove Corollary~\ref{corollary:transformed-dual-lattice}.

  \item
    Show that the successive minima and the covering radius of a lattice are well-defined,
    i.e. that the minimum (respectively maximum) in the definition is achieved.

  \item
    \begin{enumerate}[(a)]
    \item Show: Every $2$-dimensional lattice has a basis $(b_1, b_2)$
      in which $\|b_1\|_2 = \lambda_1$ and $\|b_2\| = \lambda_2$.

    \item
      Consider the \emph{parity lattice}
      $\Lambda := \{ x \in \Z^d ~:~ x_1 \equiv \dots \equiv x_d \pmod{2} \}$.
      Show: For $d \geq 5$, $\Lambda$ has no basis of shortest independent vectors,
      that is, there is no basis that satisfies $\|b_j\|_2 = \lambda_j$ for all $j$.
    \end{enumerate}

  \item
    Show $\lambda_1^\star \cdot \mu \leq 2^{(d-2)/2}$ using an LLL-reduced basis.

  \item
    \begin{enumerate}[(a)]
      \item Show that
        $\lambda_1 \leq 2 \left( \frac{\det \Lambda}{V_d} \right)^{1/d}$,
        where $V_d = \frac{\pi^{d/2}}{\Gamma(\frac{d}{2} + 1)}$ is the volume of a unit ball.

      \item Show: $\lambda_1 \cdot \lambda_1^\star \leq d$.
    \end{enumerate}

  \item
    Show that lattice width is invariant under linear transformations:
    Let $\Lambda$ be a lattice, $K$ be a convex body, and $f$ an invertible linear transformation.
    Then $w(K, \Lambda) = w(f(K), f(\Lambda))$.

\end{enumerate}


% Copyright 2013 Nicolai Hähnle <nhaehnle@gmail.com>
%
% This work is licensed under the Creative Commons Attribution-ShareAlike 3.0
% Unported License, see http://creativecommons.org/licenses/by-sa/3.0/
%
% Among other things, this means that yes, you may take e.g. illustrations from
% the book and use them in your own work. However, (a) you must give proper
% attribution by naming me as its original author and (b) you must make your
% derivative work available under the same or similar license terms.
%
% See the Creative Commons website for the exact licensing terms.

\chapter{Lattice programming}





\chapter{Future chapters}
\label{chapter:not-yet}

They have not been written yet. That's why they're called \emph{future} chapters.



\bibliographystyle{alpha}
\bibliography{literature}

\end{document}

